%%
%% This is file `sample-sigconf.tex',
%% generated with the docstrip utility.
%%
%% The original source files were:
%%
%% samples.dtx  (with options: `sigconf')
%% 
%% IMPORTANT NOTICE:
%% 
%% For the copyright see the source file.
%% 
%% Any modified versions of this file must be renamed
%% with new filenames distinct from sample-sigconf.tex.
%% 
%% For distribution of the original source see the terms
%% for copying and modification in the file samples.dtx.
%% 
%% This generated file may be distributed as long as the
%% original source files, as listed above, are part of the
%% same distribution. (The sources need not necessarily be
%% in the same archive or directory.)
%%
%%
%% Commands for TeXCount
%TC:macro \cite [option:text,text]
%TC:macro \citep [option:text,text]
%TC:macro \citet [option:text,text]
%TC:envir table 0 1
%TC:envir table* 0 1
%TC:envir tabular [ignore] word
%TC:envir displaymath 0 word
%TC:envir math 0 word
%TC:envir comment 0 0
%%
%%
%% The first command in your LaTeX source must be the \documentclass
%% command.
%%
%% For submission and review of your manuscript please change the
%% command to \documentclass[manuscript, screen, review]{acmart}.
%%
%% When submitting camera ready or to TAPS, please change the command
%% to \documentclass[sigconf]{acmart} or whichever template is required
%% for your publication.
%%
%%
\documentclass[sigconf]{article}
\usepackage{graphicx}
\usepackage{hyperref}
\usepackage{listings}
\usepackage{apacite}
\usepackage{helvet}

\graphicspath{ {./images/} }
%%
%% \BibTeX command to typeset BibTeX logo in the docs
\AtBeginDocument{%
  \providecommand\BibTeX{{%
    Bib\TeX}}}

%% Rights management information.  This information is sent to you
%% when you complete the rights form.  These commands have SAMPLE
%% values in them; it is your responsibility as an author to replace
%% the commands and values with those provided to you when you
%% complete the rights form.
%%\setcopyright{acmcopyright}
%%\copyrightyear{2022}
%%\acmYear{2022}
%%\acmDOI{XXXXXXX.XXXXXXX}

%% These commands are for a PROCEEDINGS abstract or paper.
%%\acmConference[]{}{December,
%%  2022}{Vancouver, WA}
%%
%%  Uncomment \acmBooktitle if the title of the proceedings is different
%%  from ``Proceedings of ...''!
%%
%%\acmBooktitle{Woodstock '18: ACM Symposium on Neural Gaze Detection,
%%  June 03--05, 2018, Woodstock, NY}
%%\acmPrice{15.00}
%%\acmISBN{978-1-4503-XXXX-X/18/06}


%%
%% Submission ID.
%% Use this when submitting an article to a sponsored event. You'll
%% receive a unique submission ID from the organizers
%% of the event, and this ID should be used as the parameter to this command.
%%\acmSubmissionID{123-A56-BU3}

%%
%% For managing citations, it is recommended to use bibliography
%% files in BibTeX format.
%%
%% You can then either use BibTeX with the ACM-Reference-Format style,
%% or BibLaTeX with the acmnumeric or acmauthoryear sytles, that include
%% support for advanced citation of software artefact from the
%% biblatex-software package, also separately available on CTAN.
%%
%% Look at the sample-*-biblatex.tex files for templates showcasing
%% the biblatex styles.
%%

%%
%% The majority of ACM publications use numbered citations and
%% references.  The command \citestyle{authoryear} switches to the
%% "author year" style.
%%
%% If you are preparing content for an event
%% sponsored by ACM SIGGRAPH, you must use the "author year" style of
%% citations and references.
%% Uncommenting
%% the next command will enable that style.
%%\citestyle{acmauthoryear}


%%
%% end of the preamble, start of the body of the document source.
\begin{document}

%%
%% The "title" command has an optional parameter,
%% allowing the author to define a "short title" to be used in page headers.
\title{Lab 1}
\author{
  \bold Ian Manix \\\
  }
%%\author{Alexander Kochenkov}
%%\author{Kevin Sun}
%%\author{Trace Logan}

%%
%% The "author" command and its associated commands are used to define
%% the authors and their affiliations.
%% Of note is the shared affiliation of the first two authors, and the
%% "authornote" and "authornotemark" commands
%% used to denote shared contribution to the research.

%%
%% By default, the full list of authors will be used in the page
%% headers. Often, this list is too long, and will overlap
%% other information printed in the page headers. This command allows
%% the author to define a more concise list
%% of authors' names for this purpose.

%%
%% The abstract is a short summary of the work to be presented in the
%% article.


%%
%% The code below is generated by the tool at http://dl.acm.org/ccs.cfm.
%% Please copy and paste the code instead of the example below.
%%

%%
%% Keywords. The author(s) should pick words that accurately describe
%% the work being presented. Separate the keywords with commas.
%%\keywords{}
%% A "teaser" image appears between the author and affiliation
%% information and the body of the document, and typically spans the
%% page.

%%\received{28 November 2022}
%%\received[revised]{N/A}
%%\received[accepted]{N/A}

%%
%% This command processes the author and affiliation and title
%% information and builds the first part of the formatted document.
%%\maketitle
%% \begin{CS 466}
 %%LABRATORY REPORT
 %%\end{CS 466}
\begin{center}
   \textbf{Hardware, Development Tools, and Blinking LED}

   \vspace{.75in}

   by

   \textbf{Ian Manix}

   \vspace{2in}

   CS 466

   LABRATORY REPORT

   \vspace{.75in}

   Computer Science, Washington State University

   January 20, 2023

\end{center}

\section{Objective}

The objectives of this lab were to 
\begin{enumerate}
  \item To setup the environment used to program the Pi pico
  \item To successfully compile and flash test code
  \item To interact with the Pi pico via gpio in and out
  \item To use the oscilloscope to analyze what the Pi pico is doing
\end{enumerate}

\clearpage
\section{Apparatus}

For this lab we used:
\begin{enumerate}
  \item 1 raspberry Pi pico 
  \item 1 breadboard
  \item 1 oscilloscope
  \item 2 push button switches
  \item 8 jumper cables
\end{enumerate}

This is the pi setup:

\includegraphics[scale=0.08]{images/setup.jpg}

\clearpage
\section{Method}

The goal in this lab was to modify the lab1.c sample code provided to achieve 5 goals:
\begin{itemize}
  \item To mirror the inbuilt LED's behavior to GPIO 18
  \item To track the inbuilt LED's behavior on the oscilloscope
  \item To cause the LED to blink 10 times rapidly when an input is read on GPIO pin 17
  \item To cause the LED to blink 20 times rapidly when an input is read on GPIO pin 18
  \item To change the behavior of the LED so that it blinks more quickly while input is read on GPIO pins 17 and 18
\end{itemize}

To achieve this first the function to control the LED was modified so that whenever the LED pin was enabled so was GPIO 18. From here I was able to attach the probe of the oscilloscope to pin 18 and the clip to ground. I then calibrated the oscilloscope so the time period displayed was measured in seconds and the voltage range was between 0 and 5v. This put the display within the expected scale of the output of pin 18 assuming it was properly matching the behavior of the LED.

To detect input GPIOs 16 and 17 were enabled and tracked by a series of cascading if statements inside a loop. When input was detected at the start of the loop the desired behavior was triggered, otherwise it defaulted to the original blinking pattern in the provided code. Below is the code used:

\begin{lstlisting}[language=c,frame=single,breaklines]
/**
 * @brief CS466 Lab1 Blink program based on pico blink example
 * 
 * Copyright (c) 2020 Raspberry Pi (Trading) Ltd.
 *
 * SPDX-License-Identifier: BSD-3-Clause
 */

#define LED_PIN 25
#define SWITCH_1 16
#define SWITCH_2 17
#define OUT 18

#include "pico/stdlib.h"

void my_gpio_init(void)
{
    gpio_init(LED_PIN);
    gpio_set_dir(LED_PIN, GPIO_OUT);

    gpio_init(OUT);
    gpio_set_dir(OUT, GPIO_OUT);

    gpio_init(SWITCH_1);
    gpio_init(SWITCH_2);
    gpio_set_dir(SWITCH_1, GPIO_IN);
    gpio_set_dir(SWITCH_2, GPIO_IN);
    gpio_pull_up(SWITCH_1);
    gpio_pull_up(SWITCH_2);
}

bool led_control(bool isOn) {
    if (!isOn) {
        gpio_put(LED_PIN, 1);
        gpio_put(OUT, 1);
        return 1;
    } else {
        gpio_put(LED_PIN, 0);
        gpio_put(OUT, 0);
        return 0;
    }
}

int main() 
{
    my_gpio_init();
    bool isOn=0;
    int s1, s2;

    while (true) {
        // get pin states
        s1 = gpio_get(SWITCH_1);
        s2 = gpio_get(SWITCH_2);

        if (!s1 && s2) {
            // switch one down
            for (int i=0; i<20; i++) {
                isOn = led_control(isOn);
                sleep_ms(10);
                isOn = led_control(isOn);
                sleep_ms(56);
            }
        } else if (s1 && !s2) {
            // switch two down
            for (int i=0; i<10; i++) {
                isOn = led_control(isOn);
                sleep_ms(10);
                isOn = led_control(isOn);
                sleep_ms(66);
            }
        } else if (!s1 && !s2) {
            // both switches down
            while (!s1 && !s2) {
                // both are held
                isOn = led_control(isOn);
                sleep_ms(50);
                isOn = led_control(isOn);
                sleep_ms(150);
                
                s1 = gpio_get(SWITCH_1);
                s2 = gpio_get(SWITCH_2);
            }
        } else {
            // no buttons down
            isOn = led_control(isOn);
            sleep_ms(100);
            isOn = led_control(isOn);
            sleep_ms(900);
        }
    }
}

\end{lstlisting}

\clearpage
\section{Data}

Oscilloscope output:

\includegraphics[scale=0.08]{images/scope.jpg}

This was measured by connecting the oscilloscope clamp to the ground pin and the probe to the GPIO 18 (pin 24) via jumper cables. 

\clearpage
\section{Results and Analysis}

The expected behavior of the provided code lab1.c is that when uploaded and run, the Pi nano will blink it's onboard LED on for 100mn then off for 900ms while the Pi is connected to power.

In the oscilloscope output you can clearly see that the voltage is high for 100ms, then low for 900ms. This clearly follows the pattern of the LED, with the high and low syncing with the LED being on and off. This shows that the code was correctly outputting high on pin 24 when the led was on, and low otherwise.


\section{Conclusion}

The objectives of this lab were:

\begin{itemize}
  \item To mirror the inbuilt LED's behavior to GPIO 18
  \item To track the inbuilt LED's behavior on the oscilloscope
  \item To cause the LED to blink 10 times rapidly when an input is read on GPIO pin 17
  \item To cause the LED to blink 20 times rapidly when an input is read on GPIO pin 18
  \item To change the behavior of the LED so that it blinks more quickly while input is read on GPIO pins 17 and 18
\end{itemize}

These objectives were all met. Objectives one and two can be seen by the above oscilloscope data, as to achieve 2, 1 must have been successful. The LED behaviors described in objectives 3-5 were also achieved, which can be verified by running the provided code. The goal of reading input and outputting information over GPIO pins was achieved, as was the goal of setting up the tools needed for this and future labs.


%%
%% The next two lines define the bibliography style to be used, and
%% the bibliography file.
\bibliographystyle{ACM-Reference-Format}
\bibliography{sample-base}

\end{document}
\endinput
%%
%% End of file `sample-sigconf.tex'.