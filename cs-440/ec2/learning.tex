
\documentclass{article}
\usepackage{graphicx}
\usepackage{hyperref}
\usepackage{listings}
%%\usepackage{apacite}
\usepackage{helvet}
\usepackage{ragged2e}
\usepackage{titlesec}
\usepackage{courier}
%\usepackage{fontenc}
%\usepackage{fontspec}
\usepackage{amsmath}
%\usepackage{enumitem}
\usepackage[shortlabels]{enumitem}


\lstset{basicstyle=\footnotesize\ttfamily, language=SQL}

\graphicspath{ {./images/} }

\begin{document}

    \begin{FlushLeft}
        Ian Manix

        Uncertainty

        April 26, 2023
    \end{FlushLeft}


    \section*{1}

    \begin{enumerate}[a)]
        \item   (0,0):0 - 0+0+0.05 = 0.05 -> 1 ER=-1, bias -= 0.05 \\
                (0,1):0 - 0+0.1+0  = 0.05 -> 1 ER=-1, bias -= 0.05 \\
                (1,0):0 - (-0.1)+0+(-0.05) = -0.15 -> 0 ER=0, bias unchanged \\
                (1,1):1 - (-0.1)+0.1+(-0.05) = -0.05 -> 0 ER=1, bias += 0.05 \\
                resulting bias is 0

        \item   No, weights of 0.09 and 0.09 would work: \\
                (0,0):0 = 0.05 -> 1, bias -= 0.05 \\
                (1,0):0 = 0.04 -> 1, bias -= 0.05 \\
                (1,0):0 = -0.01 -> 0, bias unchanged \\
                (1,1):1 = -0.23 -> 0, bias unchanged \\
                resulting bias is -0.05

        \item   No, there are a verity of combinations that would result in a similar outcome, as long as each weight is 0.1 > w > 0.05 they will correctly train to represent logical AND with the other specified parameters. 
    \end{enumerate}


    \section*{2}

    \begin{enumerate}[a)]
        \item   (3,3): 0.807 \\
                (3,2): 0.691 \\
                (3,1): 0.617 \\
                (4,1): 0.355
        \item   (3,3): move up \\
                (3,2): move left \\
                (3,1): move up \\
                (4,1): move down
    \end{enumerate}


\end{document}
\endinput