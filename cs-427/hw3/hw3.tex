\documentclass[sigconf]{article}
\usepackage{graphicx}
\usepackage{hyperref}
\usepackage{listings}
\usepackage{helvet}
\usepackage{ragged2e}
\usepackage{tikz}
\usepackage{titlesec}
\usepackage{courier}
\usepackage{pdfpages}


\usetikzlibrary{er,positioning, arrows.meta}

\lstset{basicstyle=\footnotesize\ttfamily, language=Java}



\graphicspath{ {./images/} }
%%
%% \BibTeX command to typeset BibTeX logo in the docs
%\AtBeginDocument{%
%  \providecommand\BibTeX{{%
%    Bib\TeX}}}

\titleformat{\section}
  {\normalfont\Large\bfseries}
  {\thesection}
  {1em}
  {}
  [{\titlerule[0.8pt]}]

\titleformat{\subsection}
  {\normalfont\Large\bfseries}
  {\thesection}
  {1em}
  {}
  [{\titlerule[0.3pt]}]

  \titleformat{\title}
  {\normalfont\Large\bfseries}
  {\thesection}
  {1em}
  {{\titlerule[0.8pt]}}
  [{\titlerule[0.8pt]}]


\title{
  %\line(1,0){250} \\
  \textbf{Homework 3} \\
  %\large \textbf{A game review} \\
  %\line(1,0){250}
  }
\author{ 
  Ian Manix
  }


%\renewcommand*\contentsname{Table of Contents}

\begin{document}

%%
%% The "title" command has an optional parameter,
%% allowing the author to define a "short title" to be used in page headers.



\maketitle

%\clearpage



%\clearpage
%\addcontentsline{toc}{section}{Concept}
\section*{}
\begin{enumerate}
  \item \begin{enumerate}
    \item 34
    \item 15
  \end{enumerate}
  \item \begin{enumerate}
    \item 11
    \item 6
    \item 3239 % not 309 or 1082
  \end{enumerate}
  \item \begin{enumerate}
    \item 1, 3, 5, 7, 9, 11, 13, 15
    \item 1, 11, 13, 7, 9, 2, 5, 15
    \item There is a very small space of potential values, making it very easy to brute force
  \end{enumerate}
  \item (37-1)(79-1) = 2808
  \item \begin{enumerate}
    \item $-x^2+7x-3$
    \item $30x^4+5x^3+27x^2+2x+6$
  \end{enumerate}
  \item \begin{enumerate}
    \item $\{0, 1, x, x+1, x^2, x^2+1, x^2+x, x^2+x+1, x^3, x^3+1, x^3+x, x^3+x+1, x^3+x^2, x^3+x^2+1, x^3+x^2+x, x^3+x^2+x+1\}$
    \item x+1
  \end{enumerate}
\end{enumerate}


%%
%% The next two lines define the bibliography style to be used, and
%% the bibliography file.
\bibliographystyle{ACM-Reference-Format}
\bibliography{sample-base}

\end{document}
\endinput
'.