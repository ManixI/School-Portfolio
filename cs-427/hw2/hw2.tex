
\documentclass[sigconf]{article}
\usepackage{graphicx}
\usepackage{hyperref}
\usepackage{listings}
\usepackage{helvet}
\usepackage{ragged2e}
\usepackage{tikz}
\usepackage{titlesec}
\usepackage{courier}
\usepackage{pdfpages}


\usetikzlibrary{er,positioning, arrows.meta}

\lstset{basicstyle=\footnotesize\ttfamily, language=Java}



\graphicspath{ {./images/} }
%%
%% \BibTeX command to typeset BibTeX logo in the docs
%\AtBeginDocument{%
%  \providecommand\BibTeX{{%
%    Bib\TeX}}}

\titleformat{\section}
  {\normalfont\Large\bfseries}
  {\thesection}
  {1em}
  {}
  [{\titlerule[0.8pt]}]

\titleformat{\subsection}
  {\normalfont\Large\bfseries}
  {\thesection}
  {1em}
  {}
  [{\titlerule[0.3pt]}]

  \titleformat{\title}
  {\normalfont\Large\bfseries}
  {\thesection}
  {1em}
  {{\titlerule[0.8pt]}}
  [{\titlerule[0.8pt]}]


\title{
  %\line(1,0){250} \\
  \textbf{Homework 2} \\
  %\large \textbf{A game review} \\
  %\line(1,0){250}
  }
\author{ 
  Ian Manix
  }


%\renewcommand*\contentsname{Table of Contents}

\begin{document}

%%
%% The "title" command has an optional parameter,
%% allowing the author to define a "short title" to be used in page headers.



\maketitle

%\clearpage



%\clearpage
%\addcontentsline{toc}{section}{Concept}
\section*{}
\begin{enumerate}
  \item Input: 1010 0010  Key: 01111 11101 \\
        3 5 2 7 4 10 1 8 9 6 | P10\\
        0 1 1 1 1 1 1 1 0 1 | \\
        1 1 1 1 0 1 1 0 1 1 | LS-1\\
        6 3 7 4 8 5 10 9 | P8\\
        1 1 1 1 0 0 1 1 | K1\\
        1 1 0 1 1 0 1 1 1 1 | LS-2\\
        6 3 7 4 8 5 10 9 | P8\\
        0 0 1 1 1 1 1 1 | K2\\
        \\
        Round 1:\\
        1 0 1 0 0 0 1 0 | Text\\
        2 6 3 1 4 8 5 7 | IP\\
        0 0 1 1 0 0 0 1 |\\
        4 1 2 3 2 3 4 1 | E/P\\
        1 0 0 0 0 0 1 0 |\\
        0 0 1 1 1 1 1 1 | K2\\
        1 0 1 1 1 1 0 1 | $\oplus$\\
        11 | 01 || 11 | 10 |\\
        0 0 0 0 | S\\
        2 4 3 1 | P4\\
        0 0 0 0 |\\
        0 0 1 1 | L\\
        0 0 1 1 | $\oplus$\\
        0 0 0 1 0 0 1 1 | SW\\ % sould be 0001 0011
        \\
        Round 2:\\
        0 0 0 1 0 0 1 1 | Text\\
        2 6 3 1 4 8 7 5 | IP\\
        0 0 0 0 1 1 1 0 |\\
        4 1 2 3 1 3 4 1 |E/P\\
        0 1 1 1 1 1 0 1 |\\
        1 1 1 1 0 0 1 1 | K1\\
        1 0 0 0 1 1 1 0 | $\oplus$\\
        10 | 00 || 10 | 11 |\\
        1 1 0 0 | S\\
        2 4 3 1 | P4\\
        1 0 0 1 |\\
        0 0 0 0 | L\\
        0 1 0 0 | $\oplus$\\
        1 1 1 0 0 1 0 0 | SW\\
        \\
        1 1 1 0 0 1 0 0 | Text\\
        4 1 3 5 7 2 8 6 | $IP^{-1}$\\
        0 1 1 0 0 1 0 1 | Final Text\\
        GF | ASCII
            




  \item 2 keys. Decrypting with a different key then encrypting is cryptographicly equivalent to encrypting it, but doing E-D-E instead E-E-E keeps 3-DES interoperable with single key DES systems.
  \item Because some modes do not require a separate decryption algorithm. In CFB for example because only the key bits are being encrypted and the data is only being XORed, the recipient dose not need a separate decryption algorithm to get the data using the key.
  \item No, the encryption and decryption of a block depends on not only the previous block, but all blocks before it so both can only be done sequentially and not in parallel
  \item \begin{itemize}
    \item[a.] Yes, the error propagates to all remaining blocks
    \item[b.] Every block contains errors that get worse as the blocks progress. This is a result of the avalanche effect of the algorithm
  \end{itemize}
  \item The error propagates until the corrupted bits clear the shift register (1 + number of bits / kept bits). The error propagates because previous data is used to encrypt future data, but eventually stops propagating because incorrectly encrypted data is not used to encrypt future data, only the raw data is used. This means once the algorithm reaches a point where it is no longer using the data with the error, said error stops propagating. 
\end{enumerate}


%%
%% The next two lines define the bibliography style to be used, and
%% the bibliography file.
\bibliographystyle{ACM-Reference-Format}
\bibliography{sample-base}

\end{document}
\endinput
'.