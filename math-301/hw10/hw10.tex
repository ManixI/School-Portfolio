
\documentclass[sigconf]{article}
\usepackage{savesym}
\usepackage{amsmath}
\usepackage{graphicx}
\usepackage{hyperref}
\usepackage{listings}
\usepackage{helvet}
\usepackage{ragged2e}
\usepackage{tikz}
\usepackage{titlesec}
\usepackage{courier}
\usepackage{pdfpages}
\usepackage{txfonts}
\usepackage{mathrsfs}
\usepackage{realhats}
%\usepackage{MnSymbol}


\usetikzlibrary{er,positioning, arrows.meta}

\lstset{basicstyle=\footnotesize\ttfamily, language=Java}



\graphicspath{ {./images/} }
%%
%% \BibTeX command to typeset BibTeX logo in the docs
%\AtBeginDocument{%
%  \providecommand\BibTeX{{%
%    Bib\TeX}}}

\titleformat{\section}
  {\normalfont\Large\bfseries}
  {\thesection}
  {1em}
  {}
  [{\titlerule[0.8pt]}]

\titleformat{\subsection}
  {\normalfont\Large\bfseries}
  {\thesection}
  {1em}
  {}
  [{\titlerule[0.3pt]}]

  \titleformat{\title}
  {\normalfont\Large\bfseries}
  {\thesection}
  {1em}
  {{\titlerule[0.8pt]}}
  [{\titlerule[0.8pt]}]


\title{
  %\line(1,0){250} \\
  \textbf{Homework 10} \\
  %\large \textbf{A game review} \\
  %\line(1,0){250}
  }
\author{ 
  Ian Manix
  }


%\renewcommand*\contentsname{Table of Contents}

\begin{document}

%%
%% The "title" command has an optional parameter,
%% allowing the author to define a "short title" to be used in page headers.



\maketitle

%\clearpage



%\clearpage
%\addcontentsline{toc}{section}{Concept}
\section*{11.1}
\begin{itemize}
  \item[1.] $\{(2,1),(3,2),(3,1),(4,3),(4,2),(4,1),(5,4),(5,3),(5,2),(5,1)\}$\\
          \includegraphics[scale=0.1]{images/2.jpg}
  \item[2.] $\{(1,1),(1,2),(1,3),(1,4),(1,5),(1,6),(2,2),(2,4),(2,6),(3,3),(3,6),(4,4),(5,5),(6,6)\}$\\
          \includegraphics[scale=0.1]{images/1.jpg}
\end{itemize}

\section*{11.2}
\begin{itemize}
  \item[1.] It is reflexive transitive, and symmetric.
  \item[5.] it is not reflexive as it is missing most of the points in $\mathbb{R}$. It is symmetric and transitive.
\end{itemize}

\section*{11.3}
\begin{itemize}
  \item[8.] The set is reflexive because if $x\in\mathbb{Z}$, then $x^2+x^2$ is even, thus $xRx$, so R is reflexive.\\
             The set is symmetric because for all $x,y\in\mathbb{Z}$, $x^2+y^2=y^2+x^2$ because of the properties of addition.\\
             Now assume $xRy$ and $yRz$\\
             Because the set R is even, then it follows that the above are both even\\
             Thus $x^2+y^2=2a$ and $y^2+x^2=2b$\\
             Adding them together we get $(x^2+y^2)+(y^2+z^2)=2a+2b=2(a+b)=2c$, which is even\\
             Thus R is an equivalence relation.\\
             The equivalence classes are [0] where $x,y$ are both even and [1] where $x,y$ are both odd.
  \item[11.] The set $\{x\in\mathbb{N}:x\mod 2\}$ has two equivalence classes $\{[0],[1]\}$, but is infinite as it is onto the infinite set $\mathbb{N}$, thus the statement is false.
  \item[12.] This is false. Let set $X=\{1,2,3\}$, the set $R=\{(1,1)(2,2)(3,3),(1,2),(2,1)\}$, and the set $S=\{(1,1),(2,2),(3,3),(2,3),(3,2)\}$\\
              R and S are both equivalence relations on the set X\\
              then $R\cap S=\{(1,1),(2,2),(3,3),(1,2),(2,1),(3,2),(2,3)\}$\\
              $R\cap S$ is not an equivalence relation as it is not transitive, it is missing the elements $\{(1,3),(3,1)\}$\\
              Therefor $R\cap S$ is not an equivalence relation on a set X just because sets R and S are.
\end{itemize}

\section*{11.4}
\begin{itemize}
  \item[3.] $\{[0],[1],[2],[3]\}$
\end{itemize}

% $\mathscr{P}$
% $\hat[ash]{a}$

Hammack, R. H. (2009). Book of Proof.\\ https://orion.math.iastate.edu/jdhsmith/class/BookOfProof.pdf

%%
%% The next two lines define the bibliography style to be used, and
%% the bibliography file.
\bibliographystyle{ACM-Reference-Format}
\bibliography{sample-base}

\end{document}
\endinput