
\documentclass[sigconf]{article}
\usepackage{graphicx}
\usepackage{hyperref}
\usepackage{listings}
\usepackage{helvet}
\usepackage{ragged2e}
\usepackage{tikz}
\usepackage{titlesec}
\usepackage{courier}
\usepackage{pdfpages}


\usetikzlibrary{er,positioning, arrows.meta}

\lstset{basicstyle=\footnotesize\ttfamily, language=Java}



\graphicspath{ {./images/} }
%%
%% \BibTeX command to typeset BibTeX logo in the docs
%\AtBeginDocument{%
%  \providecommand\BibTeX{{%
%    Bib\TeX}}}

\titleformat{\section}
  {\normalfont\Large\bfseries}
  {\thesection}
  {1em}
  {}
  [{\titlerule[0.8pt]}]

\titleformat{\subsection}
  {\normalfont\Large\bfseries}
  {\thesection}
  {1em}
  {}
  [{\titlerule[0.3pt]}]

  \titleformat{\title}
  {\normalfont\Large\bfseries}
  {\thesection}
  {1em}
  {{\titlerule[0.8pt]}}
  [{\titlerule[0.8pt]}]


\title{
  %\line(1,0){250} \\
  \textbf{Homework 1} \\
  %\large \textbf{A game review} \\
  %\line(1,0){250}
  }
\author{ 
  Ian Manix
  }


%\renewcommand*\contentsname{Table of Contents}

\begin{document}

%%
%% The "title" command has an optional parameter,
%% allowing the author to define a "short title" to be used in page headers.



\maketitle

%\clearpage



%\clearpage
%\addcontentsline{toc}{section}{Concept}
\section*{2.1}
\subsection*{1}
This is a statement, it is false

\subsection*{5}
This is a statement, it is true

\clearpage
\section*{2.2}
\subsection*{1}
P: 8 is even \\
Q: * is a power of 2 \\
\\
$P \land Q$

\subsection*{3}
P: X == Y \\
\\
$\neg P$

\subsection*{9}
$x \in A$ \\
$x \in B$ \\
\\
$x \in A \land x \not\in B$

\clearpage
\section*{2.3}
\subsection*{2}
If a function is differentiable then it is is continuous

\subsection*{3}
Only if a function is integrable then is it continuous

\subsection*{11}
If you stop writing you fail.


\clearpage
\section*{2.4}
\subsection*{2}
If and only if a function is linear then dose it have a constant derivative.


\clearpage
\section*{2.5}
\subsection*{1}
\begin{center}
\begin{tabular}{ |c|c|c|c|c| } 
\hline
P & Q & R & $Q \Rightarrow R$ & $P \lor (Q \Rightarrow R)$ \\ 
\hline
T & T & T & T & T \\
\hline
T & T & F & F & T \\
\hline
T & F & T & T & T \\
\hline
T & F & F & T & T \\
\hline
F & T & T & T & T \\
\hline
F & T & F & F & F \\
\hline
F & F & T & T & T \\
\hline
F & F & F & T & T \\
\hline
\end{tabular}
\end{center}

\subsection*{4}
\begin{center}
\begin{tabular}{ |c|c|c|c|c| } 
\hline
P & Q & $P \lor Q$ & $\neg P$ & $\neg (P \lor Q) \lor (\neg P)$ \\ 
\hline
T & T & T & F & F \\
\hline
T & F & T & F & F \\
\hline
F & T & T & T & T \\
\hline
F & F & F & T & T \\
\hline
\end{tabular}
\end{center}
%%
%% The next two lines define the bibliography style to be used, and
%% the bibliography file.
\bibliographystyle{ACM-Reference-Format}
\bibliography{sample-base}

\end{document}
\endinput