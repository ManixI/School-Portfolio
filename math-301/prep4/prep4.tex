
\documentclass[sigconf]{article}
\usepackage{graphicx}
\usepackage{hyperref}
\usepackage{listings}
\usepackage{helvet}
\usepackage{ragged2e}
\usepackage{tikz}
\usepackage{titlesec}
\usepackage{courier}
\usepackage{pdfpages}
\usepackage{txfonts}


\usetikzlibrary{er,positioning, arrows.meta}

\lstset{basicstyle=\footnotesize\ttfamily, language=Java}



\graphicspath{ {./images/} }
%%
%% \BibTeX command to typeset BibTeX logo in the docs
%\AtBeginDocument{%
%  \providecommand\BibTeX{{%
%    Bib\TeX}}}

\titleformat{\section}
  {\normalfont\Large\bfseries}
  {\thesection}
  {1em}
  {}
  [{\titlerule[0.8pt]}]

\titleformat{\subsection}
  {\normalfont\Large\bfseries}
  {\thesection}
  {1em}
  {}
  [{\titlerule[0.3pt]}]

  \titleformat{\title}
  {\normalfont\Large\bfseries}
  {\thesection}
  {1em}
  {{\titlerule[0.8pt]}}
  [{\titlerule[0.8pt]}]


\title{
  %\line(1,0){250} \\
  \textbf{Preparation 4} \\
  %\large \textbf{A game review} \\
  %\line(1,0){250}
  }
\author{ 
  Ian Manix
  }


%\renewcommand*\contentsname{Table of Contents}

\begin{document}

%%
%% The "title" command has an optional parameter,
%% allowing the author to define a "short title" to be used in page headers.



\maketitle

%\clearpage



%\clearpage
%\addcontentsline{toc}{section}{Concept}

\begin{enumerate}
  \item A statement that is true and has been proven to be true
  \item Definitions need to be unambiguous and simple enough that they can be easily build upon. Wordy definitions make this more difficult
  \item \textbf{Proposition:} if a and b are even and c = a + b, then c is even.\\
        Suppose a and b are even \\
        Then an integer n is even if n = 2a for some integer $a \ni \mathbb{Z}$\\
        Thus $a = 2x$ and $b = 2y$ where $x,y \ni \mathbb{Z}$\\
        So $c = 2(x+y)$\\
        Then, if $a,b \ni \mathbb{Z}$, then $a+b \ni \mathbb{Z}, a - b \ni \mathbb{Z}$, and $ab \ni \mathbb{Z}$\\
        Thus in a = x + y, $a \ni \mathbb{Z}$ if $x,y \ni \mathbb{Z}$\\
        Thus c = 2a \\
        Therefore, c is even
\end{enumerate}

%%
%% The next two lines define the bibliography style to be used, and
%% the bibliography file.
\bibliographystyle{ACM-Reference-Format}
\bibliography{sample-base}

\end{document}
\endinput