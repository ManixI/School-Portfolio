
\documentclass[sigconf]{article}
\usepackage{savesym}
\usepackage{amsmath}
\usepackage{graphicx}
\usepackage{hyperref}
\usepackage{listings}
\usepackage{helvet}
\usepackage{ragged2e}
\usepackage{tikz}
\usepackage{titlesec}
\usepackage{courier}
\usepackage{pdfpages}
\usepackage{txfonts}
\usepackage{mathrsfs}
\usepackage{realhats}
%\usepackage{MnSymbol}


\usetikzlibrary{er,positioning, arrows.meta}

\lstset{basicstyle=\footnotesize\ttfamily, language=Java}



\graphicspath{ {./images/} }
%%
%% \BibTeX command to typeset BibTeX logo in the docs
%\AtBeginDocument{%
%  \providecommand\BibTeX{{%
%    Bib\TeX}}}

\titleformat{\section}
  {\normalfont\Large\bfseries}
  {\thesection}
  {1em}
  {}
  [{\titlerule[0.8pt]}]

\titleformat{\subsection}
  {\normalfont\Large\bfseries}
  {\thesection}
  {1em}
  {}
  [{\titlerule[0.3pt]}]

  \titleformat{\title}
  {\normalfont\Large\bfseries}
  {\thesection}
  {1em}
  {{\titlerule[0.8pt]}}
  [{\titlerule[0.8pt]}]


\title{
  %\line(1,0){250} \\
  \textbf{Homework 7} \\
  %\large \textbf{A game review} \\
  %\line(1,0){250}
  }
\author{ 
  Ian Manix
  }


%\renewcommand*\contentsname{Table of Contents}

\begin{document}

%%
%% The "title" command has an optional parameter,
%% allowing the author to define a "short title" to be used in page headers.



\maketitle

%\clearpage



%\clearpage
%\addcontentsline{toc}{section}{Concept}
\section*{Chapter 8}
\begin{itemize}
  \item[1.] Prove that $\{12n:n\in\mathbb{Z}\}\subseteq\{2n:n\in\mathbb{Z}\}\cap\{3n:3\in\mathbb{Z}\}$\\
            Suppose $a\in\{2n:n\in\mathbb{Z}\}\cap\{3n:3\in\mathbb{Z}\}$\\
            This means $a\in\{3n:3\in\mathbb{Z}\}$ and $a\in\{2n:3\in\mathbb{Z}\}$ by definition of intersection\\
            This means $2|a$ and $a|3$, so $a=2c$ and $a=3d$ for $c,d\in\mathbb{Z}$\\
            Because $a=2c$, a is even as this is the definition of an even number\\
            Thus $d$ is even, or $d=2e$ for $e\in\mathbb{Z}$\\
            So $a=3(2e)=6e$, because a is still even $e=2f$ or $a=12f$\\
            Thus $12|a$ ie. $\{12n:n\in\mathbb{Z}\}$\\
            \hspace*{5mm} Because $a\in\{2n:n\in\mathbb{Z}\}\cap\{3n:3\in\mathbb{Z}\}$ implies $12|a$, $\{12n:n\in\mathbb{Z}\}$, it must be true that $\{12n:n\in\mathbb{Z}\}\subseteq\{2n:n\in\mathbb{Z}\}\cap\{3n:3\in\mathbb{Z}\}$


  \item[2.] Prove that $\{6n:n\in\mathbb{Z}\}=\{2n:n\in\mathbb{Z}\}\cap\{3n:3\in\mathbb{Z}\}$\\
            Suppose $a\{6n:n\in\mathbb{Z}\}$\\
            This means $6|a$ or $a=6c$ for $a\in\mathbb{Z}$\\
            Therefor $a=2(3c)$ so it follows that $3|a$ and $2|a$\\
            So $a\in\{3n:3\in\mathbb{Z}\}$ and $a\in\{2n:3\in\mathbb{Z}\}$\\
            Therefor $a\in\{2n:n\in\mathbb{Z}\}\cap\{3n:3\in\mathbb{Z}\}$\\
            This shows $\{2n:n\in\mathbb{Z}\}\cap\{3n:3\in\mathbb{Z}\}\subseteq \{12n:n\in\mathbb{Z}\}$\\
            Now suppose $a\in\{2n:n\in\mathbb{Z}\}\cap\{3n:3\in\mathbb{Z}\}$\\
            This means $a\in\{3n:3\in\mathbb{Z}\}$ and $a\in\{2n:3\in\mathbb{Z}\}$ by definition of intersection\\
            This means $2|a$ and $a|3$, so $a=2c$ and $a=3d$ for $c,d\in\mathbb{Z}$\\
            Thus a has both 2 and 3 as prime factors, therefor the prime factorization of a must include 2 and 3\\
            Hence $2*3=6$ divides a, ie. $a\{6n:n\in\mathbb{Z}\}$\\
            Therefor $\{2n:n\in\mathbb{Z}\}\cap\{3n:3\in\mathbb{Z}\}\subseteq a\{6n:n\in\mathbb{Z}\}$\\
            \hspace*{5mm}Because $\{2n:n\in\mathbb{Z}\}\cap\{3n:3\in\mathbb{Z}\}\subseteq \{12n:n\in\mathbb{Z}\}$ and $\{2n:n\in\mathbb{Z}\}\cap\{3n:3\in\mathbb{Z}\}\subseteq a\{6n:n\in\mathbb{Z}\}$, $\{6n:n\in\mathbb{Z}\}=\{2n:n\in\mathbb{Z}\}\cap\{3n:3\in\mathbb{Z}\}$



  \item[3.] If $k\in\mathbb{Z}$, then $\{x\in\mathbb{Z}:n|k\}\subseteq\{n\in\mathbb{Z}:n|k^2\}$\\
            Let $n\in\{n\in\mathbb{Z}:n|k^2\}$\\
            Then $n=k^2=k(k)$\\
            Thus if $n|kk$, $n|k$\\
            Therefor $n\in\{x\in\mathbb{Z}:n|k\}$\\
            Because $n\in\{n\in\mathbb{Z}:n|k^2\}\Rightarrow n\in\{x\in\mathbb{Z}:n|k\}$, $\{x\in\mathbb{Z}:n|k\}\subseteq\{n\in\mathbb{Z}:n|k^2\}$

  \item[6.] Suppose A, B, and C are sets. Prove that if $A\subseteq B$, then $A-C\subseteq B-C$\\
            Suppose A, B, and C are sets and $A\subseteq B$\\
            Then if we can subtract C from both sides to get $A-C\subseteq B-C$\\
            \hspace*{5mm} This works because by removing the same elements of C from both sets A and B, you can never reach a point where B dose not contain every element of A.


  \item[8.] If A, B, and C are sets, then $A\cup(B\cap C)=(A\cup B)\cap(A\cup C)$\\
            Suppose $A\cup(B\cap C)$\\
            This means $\{x:(x\in A)\lor((x\in B)\land(x\in C))\}$\\
            Then $\{x:((x\in A)\land(x\in B))\lor((x\in A)\land(x\in C))$ by the Distributive law\\
            So $A\cup(B\cap C)=(A\cup B)\cap(A\cup C)$


  \item[20.] Prove that $\{9^n:n\in\mathbb{Q}\}=\{3^n:n\in\mathbb{Q}\}$\\
            Suppose $x\in \{9^n:n\in\mathbb{Q}\}$\\
            Then $9^n$ can be written as $3^{n+1}$\\
            And $\mathbb{Q}$ is closed under addition by theorem 1\\
            Therefor $\{a:a\in\mathbb{Q}\}=\{a+1:a\in\mathbb{Q}\}$\\
            Thus $\{3^n:n\in\mathbb{Q}\}=\{3^{n+1}:n\in\mathbb{Q}\}=\{9^n:n\in\mathbb{Q}\}$


\end{itemize}

% $\mathscr{P}$
% $\hat[ash]{a}$

Hammack, R. H. (2009). Book of Proof.\\ https://orion.math.iastate.edu/jdhsmith/class/BookOfProof.pdf

%%
%% The next two lines define the bibliography style to be used, and
%% the bibliography file.
\bibliographystyle{ACM-Reference-Format}
\bibliography{sample-base}

\end{document}
\endinput