
\documentclass[sigconf]{article}
\usepackage{graphicx}
\usepackage{hyperref}
\usepackage{listings}
\usepackage{helvet}
\usepackage{ragged2e}
\usepackage{tikz}
\usepackage{titlesec}
\usepackage{courier}
\usepackage{pdfpages}
\usepackage{txfonts}


\usetikzlibrary{er,positioning, arrows.meta}

\lstset{basicstyle=\footnotesize\ttfamily, language=Java}



\graphicspath{ {./images/} }
%%
%% \BibTeX command to typeset BibTeX logo in the docs
%\AtBeginDocument{%
%  \providecommand\BibTeX{{%
%    Bib\TeX}}}

\titleformat{\section}
  {\normalfont\Large\bfseries}
  {\thesection}
  {1em}
  {}
  [{\titlerule[0.8pt]}]

\titleformat{\subsection}
  {\normalfont\Large\bfseries}
  {\thesection}
  {1em}
  {}
  [{\titlerule[0.3pt]}]

  \titleformat{\title}
  {\normalfont\Large\bfseries}
  {\thesection}
  {1em}
  {{\titlerule[0.8pt]}}
  [{\titlerule[0.8pt]}]


\title{
  %\line(1,0){250} \\
  \textbf{Homework 5} \\
  %\large \textbf{A game review} \\
  %\line(1,0){250}
  }
\author{ 
  Ian Manix
  }


%\renewcommand*\contentsname{Table of Contents}

\begin{document}

%%
%% The "title" command has an optional parameter,
%% allowing the author to define a "short title" to be used in page headers.



\maketitle

%\clearpage



%\clearpage
%\addcontentsline{toc}{section}{Concept}

\begin{itemize}
  \item[Ch 4.] \begin{itemize}
      \item[15.]  Proposition: if $n \in \mathbb{Z}$, then $n^{2}+3n+4$ is even\\
                  Suppose $n\in\mathbb{Z}$\\
                  A number is even if $n=2x$ where $x\in\mathbb{Z}$
                  Case, if n is even\\
                  Then $n^2 = 4x^2 = 2(2x^2) = 2(k)$ where $k\in\mathbb{Z}$ because of fact 4.1\\
                  Thus $n^2$ is even\\
                  A number is odd if $n=2y+1$ where $y\in\mathbb{Z}$\\
                  thus $3n = (2y+1)(2x) = 2(xy*x) = x(k)$ where $x\in\mathbb{Z}$ because of fact 4.1\\
                  Thus $3n is even$\\
                  So $n^2+3n+4=2p+2q+2r=2(p+q+r)=2(s)$ where $p,q,r,s\in\mathbb{Z}$ because of fact 4.1\\
                  Thus $n^{2}+3n+4$ is even if n is even\\
                  Case, if n is odd\\
                  A number is odd if $n=2x+1$ where $x\in\mathbb{Z}$\\
                  Then $n^2=(2x+1)^2=4x^2+4x+1=2(2x^2+2x)+1=2k+1$ where $x,k\in\mathbb{Z}$ because of fact 4.1\\
                  Thus $n^2$ is odd\\
                  And $3n=(2y+1)(2x+1)=4xy+2x+2y+1=2(2xy+x+y)+1=2k+1$ where $x,y,k\in\mathbb{Z}$ because of fact 4.1\\
                  Thus $3n is odd$\\
                  So $n^2+3n+4=(2x+1)+(2y+1)+2z=2x+2y+2z+2=2(x+y+z+1)=2w$ where $w,x,y,z\in\mathbb{Z}$\\
                  Thus $n^{2}+3n+4$ is even if n is odd\\
                  Thus $n^{2}+3n+4$ is even in all cases

      \item[16.]  Proposition: if two integers have the same parity, then their sum is even\\
                  Suppose $x,y\in\mathbb{Z}$ and $x,y$ have the same parity\\
                  Case, if x and y are even\\
                  A number is even if $n=2x$ where $x\in\mathbb{Z}$
                  Then $x+y=(2p)+(2q)=2(p+q)=2r$ where $p,q,r\in\mathbb{Z}$ because of fact 4.1\\
                  Thus $x+y$ is even if x and y are even\\
                  Case, if x and y are odd\\
                  A number is odd if $n=2x+1$ where $x\in\mathbb{Z}$\\
                  Then $x+y=(2p+1)+(2q+1)=2p+2q+2=2(p+q+1)=2r$ where $p,q,r\in\mathbb{Z}$ because of fact 4.1\\
                  Thus $x+y$ is even if x and y are odd\\
                  Thus $x+y$ is even in all cases where x and y have the same polarity
    \end{itemize}
  \item[Ch 5.] \begin{itemize}
      \item[19.]  Proposition: let $a,b,c\in \mathbb{Z}$ and $n\in\mathbb{N}$. If $a\equiv b$ (mod n) and $a\equiv c$ (mod n), than $a\equiv c$ (mod n)\\
                  Suppose $a,b,c\in\mathbb{Z}$, $n\in\mathbb{N}$, $a\equiv b$, and $a\equiv c$\\
                  Then $n|(a-b)$ and $n|(b-c)$ because of definition 4.4\\
                  Thus $a-b=nx$ and $b-c=ny$ where and $x,y\in]mathbb{Z}$\\
                  Then combining them gives us $c-b=nd-ne=n(d-e)$\\
                  So $n|(c-b)$\\
                  Thus $c\equiv b (mod n)$ by definition 5.1


      \item[21.]  Proposition: let $a,b\in\mathbb{Z}$ and $n\in\mathbb{N}$. If $a\equiv b$ (mod n), then $a^3 \equiv b^3$ (mod n)\\
                  Suppose $a\equiv b mod n$\\
                  Then $n|(a-b)$ and there exists $c$ where $a-b=nc$ where $c\in\mathbb{Z}$ because \\
                  So if $a-b = nc$ then: \\
                  $(a-b)=(nc)$\\
                  $(a-b)(a^2+ab+b^2)=nc(a^2+ab+b^2)$\\
                  $a^3+a^2b+ab^2-a^2b-ab^2-b^3=nc(a^2+ab+b^2)$\\
                  $a^3-b^3=nc(a^2+ab+b^2)$\\
                  Then $a^2+ab+b^2\in\mathbb{Z}$ because of fact 4.1 \\
                  And $n|(a^3-b^3)$ because of definition 4.4\\
                  Thus $a^3 \equiv b^3$ (mod n)

    \end{itemize}
  \item[Ch 7.] \begin{itemize}
      \item[1.]   Proposition: suppose $x\in\mathbb{Z}$. Then $x$ is even and if and only if $3x+5$ is odd\\
                  Suppose $x\in\mathbb{Z}$ and $x$ is even\\
                  A number is even if $x=2n$ where $n\in\mathbb{Z}$\\
                  Thus $3x+5=x+2(x+2)+1$\\
                  And because we can say $x=2n$, $x+2(x+2)+1=2n+2(2n+2)+1=2(3n+2)+1=2m+1$ where $m=2n+2$\\
                  So if $x$ is even then $3x+5$ is odd\\
                  Now suppose $x$ is odd\\
                  A number is odd if $x=2n+1$ where $n\in\mathbb{Z}$\\
                  So $3x+5=3(2n+1)+5=6n+8=2(3n+4)=2m$ where $m\in\mathbb{Z}$\
                  Thus if $x$ is odd $3x+5$ is even\\
                  Showing $x$ is even if and only if $3x+5$ is odd

      \item[17.]  Proposition: There is a prime number between 90 and 100\\ %97
                  97 is a number between 90 and 100 that is prime\\
                  Thus the proposition is true

      \item[20.]  Proposition: There exists an $n\in\mathbb{N}$ for which $11|(2^{n}-1)$\\
                  $n=10$ is a solution where $11|(2^{n}-1)$
    \end{itemize}
\end{itemize}

Hammack, R. H. (2009). Book of Proof. https://orion.math.iastate.edu/jdhsmith/class/BookOfProof.pdf

%%
%% The next two lines define the bibliography style to be used, and
%% the bibliography file.
\bibliographystyle{ACM-Reference-Format}
\bibliography{sample-base}

\end{document}
\endinput