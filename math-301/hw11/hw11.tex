
\documentclass[sigconf]{article}
\usepackage{savesym}
\usepackage{amsmath}
\usepackage{graphicx}
\usepackage{hyperref}
\usepackage{listings}
\usepackage{helvet}
\usepackage{ragged2e}
\usepackage{tikz}
\usepackage{titlesec}
\usepackage{courier}
\usepackage{pdfpages}
\usepackage{txfonts}
\usepackage{mathrsfs}
\usepackage{realhats}
%\usepackage{MnSymbol}


\usetikzlibrary{er,positioning, arrows.meta}

\lstset{basicstyle=\footnotesize\ttfamily, language=Java}



\graphicspath{ {./images/} }
%%
%% \BibTeX command to typeset BibTeX logo in the docs
%\AtBeginDocument{%
%  \providecommand\BibTeX{{%
%    Bib\TeX}}}

\titleformat{\section}
  {\normalfont\Large\bfseries}
  {\thesection}
  {1em}
  {}
  [{\titlerule[0.8pt]}]

\titleformat{\subsection}
  {\normalfont\Large\bfseries}
  {\thesection}
  {1em}
  {}
  [{\titlerule[0.3pt]}]

  \titleformat{\title}
  {\normalfont\Large\bfseries}
  {\thesection}
  {1em}
  {{\titlerule[0.8pt]}}
  [{\titlerule[0.8pt]}]


\title{
  %\line(1,0){250} \\
  \textbf{Homework 11} \\
  %\large \textbf{A game review} \\
  %\line(1,0){250}
  }
\author{ 
  Ian Manix
  }


%\renewcommand*\contentsname{Table of Contents}

\begin{document}

%%
%% The "title" command has an optional parameter,
%% allowing the author to define a "short title" to be used in page headers.



\maketitle

%\clearpage



%\clearpage
%\addcontentsline{toc}{section}{Concept}
\section*{12.1}
\begin{itemize}
  \item[6.] The domain and codoname are both $\mathbb{Z}$, the range is $\{4x+6:x\in\mathbb{Z}\}$. $f(10)=45$.
  \item[9.] No, a function must be one-to-one, but the value 4 in the domain could correspond to either 2 or -2 in the range
\end{itemize}

\section*{12.2}
\begin{itemize}
  \item[4.] The function is injective, but not surjective. This is becuase $2n$ will always be even by definition, so the range is missing all even values.\\
            It is injective because if we assume $f(n)=f(m)$ we get $(2n,n+3)=(2m,m+3)$ which can be reduced down to $(n,n=m,m)$.
  \item[5.] The funcion is injective because if we assume $f(n)=f(m)$ we get $2n+1=2m+1$ which reduced down to $n=m$.\\
            The function is not surjective because $f(n)$ will always be odd by definfition, and thus there are values in $\mathbb{Z}$ that are not in the range.
  \item[6.] This is not injective because the values (1,2) and (5,5) both return the value -5.\\
            This is surjective because $3n-4m=x$ can represent any value x for $x\in\mathbb{Z}$. This is the case as for any time $n=m$, the function $3n-4n=-n$, so as long as $n\in\mathbb{Z}$, $x\in\mathbb{Z}$.
\end{itemize}

\section*{12.3}
\begin{itemize}
  \item[4.] Take a square with sides of length 1\\
            The furthest apart any 4 points inside the square can be is 1, because that places each point at a corner of the square\\
            Then add in a 5th point
            The diagonal of this square is $l^2=1^2+1^2=2$\\
            Then $l=\sqrt{2}$\\
            If you place the 5th point the furthest away from any of the earlier 4 points it can possibly be, it ends up in the center where it is equidistant for all 4 points.\\
            This being the center of the diagonal, it is $\frac{\sqrt{2}}{2}$ from all 4 corners.\\
            Thus of 5 points all placed in a square with sides of length one, at least two will be $\frac{\sqrt{2}}{2}$ or closer to each other.
  \item[5.] Let A be a set of 7 integers and $B=\{\{0\},\{1,9\},\{2,8\},\{3,7\},\{4,6\},\{5\}\}$ where all the two element sets sum to 10.\\
            Then let $f:A\rightarrow B=x \mod 10$\\
            Because |A| > |B| f is not injunctive by the piginhole principle.\\
            Then there are two cases for any two integers x and y.
            If $x \mod 10=y\mod 10$ then $x-y$ is divisible by 10.\\
            If $x\mod 10\neq y\mod 10$ then it follows that $f(x)=f(y)=\{a,b\}$ is one of the two element sets of B.\\
            In this case we can say $x=10i+r$ and $y=10j+s$ where $i,j\in\mathbb{Z}$ and $r+s=10$.\\
            Thus $x+y=10i+10j+10=10(r+j)$ which is divisible by 10.
\end{itemize}

\section*{12.4}
\begin{itemize}
  \item[4.] $g\circ f=\{(a,a),(b,b),(c,a)\}$\\
            $f\circ g=\{(a,c),(b,c),(c,c)\}$
  \item[5.] $g\circ f=\sqrt[3]{x^3+1}$\\
            $f\circ g=x+1$
\end{itemize}

\section*{12.5}
\begin{itemize}
  \item[3.] It is injective because $2^n=2^m$ can be simplified to $m=n$.\\
            It is surjective\\
            Suppose $b\in B$\\
            Then by definition $b=2^n$ for some $n\in\mathbb{Z}$\\
            Thus $(n)=2^n=b$\\
            $f^{-1}(x)=\log_2(x)$
  \item[5.] $f^{-1}(x)=\frac{x+e}{\pi}$
\end{itemize}

\section*{12.6}
\begin{itemize}
  \item[2.] $f(\{1,2,3\})=\{3,8\}$\\
            $f(\{4,5,6,7\})=\{1,2,4,6\}$\\
            $f(\emptyset)=\emptyset$\\
            $f^{-1}(\{0,5,9\})=\emptyset$\\
            $f^{-1}(\{0,3,5,9\})=\{3\}$
\end{itemize}

% $\mathscr{P}$
% $\hat[ash]{a}$

Hammack, R. H. (2009). Book of Proof.\\ https://orion.math.iastate.edu/jdhsmith/class/BookOfProof.pdf

%%
%% The next two lines define the bibliography style to be used, and
%% the bibliography file.
\bibliographystyle{ACM-Reference-Format}
\bibliography{sample-base}

\end{document}
\endinput