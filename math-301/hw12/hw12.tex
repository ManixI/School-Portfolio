
\documentclass[sigconf]{article}
\usepackage{savesym}
\usepackage{amsmath}
\usepackage{graphicx}
\usepackage{hyperref}
\usepackage{listings}
\usepackage{helvet}
\usepackage{ragged2e}
\usepackage{tikz}
\usepackage{titlesec}
\usepackage{courier}
\usepackage{pdfpages}
\usepackage{txfonts}
\usepackage{mathrsfs}
\usepackage{realhats}
\usepackage{array}
%\usepackage{MnSymbol}


\usetikzlibrary{er,positioning, arrows.meta}

\lstset{basicstyle=\footnotesize\ttfamily, language=Java}



\graphicspath{ {./images/} }
%%
%% \BibTeX command to typeset BibTeX logo in the docs
%\AtBeginDocument{%
%  \providecommand\BibTeX{{%
%    Bib\TeX}}}

\titleformat{\section}
  {\normalfont\Large\bfseries}
  {\thesection}
  {1em}
  {}
  [{\titlerule[0.8pt]}]

\titleformat{\subsection}
  {\normalfont\Large\bfseries}
  {\thesection}
  {1em}
  {}
  [{\titlerule[0.3pt]}]

  \titleformat{\title}
  {\normalfont\Large\bfseries}
  {\thesection}
  {1em}
  {{\titlerule[0.8pt]}}
  [{\titlerule[0.8pt]}]


\title{
  %\line(1,0){250} \\
  \textbf{Homework 12} \\
  %\large \textbf{A game review} \\
  %\line(1,0){250}
  }
\author{ 
  Ian Manix
  }


%\renewcommand*\contentsname{Table of Contents}

\begin{document}

%%
%% The "title" command has an optional parameter,
%% allowing the author to define a "short title" to be used in page headers.



\maketitle

%\clearpage



%\clearpage
%\addcontentsline{toc}{section}{Concept}
\begin{enumerate}
  \item Symmetries of a triangle:\\
      \begin{tabular}{c|cccccc}
      X      & $i$    & $r$    & $r^2$ & $s$ & $sr$ & $sr^2$ \\
      \hline
      $i$    & $i$    & $r$    & $r^2$  & $s$    & $sr$   & $sr^2$ \\
      $r$    & $r$    & $r^2$  & $i$    & $sr^2$ & $s$    & $sr$ \\
      $r^2$  & $r^2$  & $i$    & $r$    & $sr$   & $sr^2$ & $s$\\
      $s$    & $s$    & $sr$   & $sr^2$ & $i$    & $r$    & $s^2$\\
      $sr$   & $sr$   & $sr^2$ & $s$    & $r^2$  & $i$    & $r$\\
      $sr^2$ & $sr^2$ & $s$    & $sr$   & $r$    & $r^2$  & $i$\\
    \end{tabular}\\
    Where $i$ represents the identity permutation, $s$ represents reflection across the axes of symmetry, $r$ represents a rotation of 120 degrees, and $r^2$ a rotation of 240 degrees.
  \item The symmetries of a triangle are a group because it is associative, there is an identity element, and there is an inverse element for every element.\\
      The identity element is $i$ which represents the triangle having neither reflected nor rotated.\\
      The fact that there is an inverse element for every element is shown in the table above by the fact that every for and every column has at least one $i$ in it.\\
      It is associative because every result is the same triangle, so no matter the order of operations the result is always the same.
  \item There are three subgroups, one consisting of only the operation $s$ and the other consisting of the operations $r$ and $r^2$, as well as the trivial subgroup. The first two subgroups are both cyclical as the first consists of only one element, and the second has the generator $r$
  \item $\mathbb{Z}$ is a subgroup of $\mathbb{R}$.\\
      The identity element of $\mathbb{R}$ is $0$ which is in $\mathbb{Z}$.\\
      Then the inverse of any element $e\in\mathbb{Z}$ is $-a$.\\
      Finally because addition is associative in $\mathbb{R}$ it is associative in $\mathbb{Z}$.
  \item Any group that is uncountable cannot be cyclic.\\
        Suppose there exists an uncountable cyclic group $A$.\\
        Thus there must exist $e\in A$ such that every element of $A$ can be written as $a^n$for $n\in\mathbb{Z}$.\\
        So $A=\{a^n:n\in\mathbb{Z}\}$.
        And because a countable union of countable sets is countable, $A$ must be countable because it is a union between 2 sets $A$ and $\mathbb{Z}$ and $\mathbb{Z}$ is countable.\\
        But $A$ was uncountable, which is a contradiction.\\
        Therefor any group that is uncountable cannot be cyclic.
\end{enumerate}



% $\mathscr{P}$
% $\hat[ash]{a}$
\subsection*{Reference:}
Hammack, R. H. (2009). Book of Proof.\\ https://orion.math.iastate.edu/jdhsmith/class/BookOfProof.pdf
\\
\\
JUDSON, T. (2023). Abstract algebra: Theory and applications. ORTHOGONAL PUBLISHING L3C.

%%
%% The next two lines define the bibliography style to be used, and
%% the bibliography file.
\bibliographystyle{ACM-Reference-Format}
\bibliography{sample-base}

\end{document}
\endinput