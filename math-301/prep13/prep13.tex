
\documentclass[sigconf]{article}
\usepackage{graphicx}
\usepackage{hyperref}
\usepackage{listings}
\usepackage{helvet}
\usepackage{ragged2e}
\usepackage{tikz}
\usepackage{titlesec}
\usepackage{courier}
\usepackage{pdfpages}
\usepackage{txfonts}
\usepackage{array}

\usetikzlibrary{er,positioning, arrows.meta}

\lstset{basicstyle=\footnotesize\ttfamily, language=Java}



\graphicspath{ {./images/} }
%%
%% \BibTeX command to typeset BibTeX logo in the docs
%\AtBeginDocument{%
%  \providecommand\BibTeX{{%
%    Bib\TeX}}}

\titleformat{\section}
  {\normalfont\Large\bfseries}
  {\thesection}
  {1em}
  {}
  [{\titlerule[0.8pt]}]

\titleformat{\subsection}
  {\normalfont\Large\bfseries}
  {\thesection}
  {1em}
  {}
  [{\titlerule[0.3pt]}]
{}
  \titleformat{\title}
  {\normalfont\Large\bfseries}
  {\thesection}
  {1em}
  {{\titlerule[0.8pt]}}
  [{\titlerule[0.8pt]}]


\title{
  %\line(1,0){250} \\
  \textbf{Preparation 13} \\
  %\large \textbf{A game review} \\
  %\line(1,0){250}
  }
\author{ 
  Ian Manix
  }


%\renewcommand*\contentsname{Table of Contents}

\begin{document}

%%
%% The "title" command has an optional parameter,
%% allowing the author to define a "short title" to be used in page headers.



\maketitle

%\clearpage



%\clearpage
%\addcontentsline{toc}{section}{Concept}

\begin{enumerate}
  \item It is a group that is closed under addition and multiplication.
  \item It must have unity, be commutative, and every non zero must be a unit.
  \item Addition: \\
    \begin{tabular}{c | ccccccccc}
      +    & 0 & 1 & 2 & i & 1+i & 2+i & 2i & 1+2i & 2+2i \\
      \hline
      0    & 0    & 1    & 2    & i    & 1+i  & 2+1  & 2i   & 1+2i & 2+2i \\
      1    & 1    & 2    & 0    & 1+i  & 2+i  & i    & 1+2i & 2+2i & 2i \\
      2    & 2    & 0    & 1    & 2+i  & i    & 1+i  & 2+2i & 2i   & 1+2i \\
      i    & i    & 1+i  & 2+i  & 0    & 2i   & 1+2i & 2    & 1    & 2 \\
      1+i  & 1+i  & 2+i  & i    & 2i   & 2    & 1    & 1+2i & 2+2i & 0 \\
      2+1  & 2+i  & i    & 1+i  & 1+2i & 0    & 2i   & 2    & 1    & 2i \\
      2i   & 2i   & 1+2i & 2+2i & 2    & 1+2i & i    & 0    & 2+i  & 1+i \\
      1+2i & 1+2i & 2+2i & 2i   & 1    & 2+2i & 2+i  & 1+i  & 0    & 2 \\
      2+2i & 2+2i & 2i   & 1+2i & 2    & 0    & 1+i  & 2+i  & 1    & i \\
    \end{tabular}\\ \\
    Multiplication:\\
    \begin{tabular}{c | ccccccccc}
      + & 0 & 1 & 2 & i & 1+i & 2+i & 2i & 1+2i & 2+2i \\
      \hline
      0    & 0    & 0    & 0    & 0    & 0    & 0    & 0    & 0    & 0 \\
      1    & 0    & 1    & 2    & i    & 1+i  & 2+1  & 2i   & 1+2i & 2+2i \\
      2    & 0    & 2    & 1    & 2i   & 2+2i & 1+2i & i    & 2+i  & 1+i \\
      i    & 0    & i    & 2i   & 2+2i & 1+2i & 2+i  & 1+i  & 2    & 1 \\
      1+i  & 0    & 1+i  & 2+2i & 1+2i & 2    & 1    & 1i   & 2i   & 1 \\
      2+1  & 0    & 2+i  & 1+2i & 2+i  & 1    & 2    & 1+i  & i    & 2 \\
      2i   & 0    & 2i   & i    & 1+i  & 1i   & 1+i  & 2    & 1+i  & 2+i \\
      1+2i & 0    & 1+2i & 2+i  & 2    & 2i   & i    & 1+i  & 2+2i & 1 \\
      2+2i & 0    & 2+2i & 1+i  & 1    & 1    & 2    & 2+i  & 1    & 2 \\
    \end{tabular}\\
    The units are: 1, 2, 1+i, 2+i, 2i, 1+2i, 2+2i
\end{enumerate}

%%
%% The next two lines define the bibliography style to be used, and
%% the bibliography file.
\bibliographystyle{ACM-Reference-Format}
\bibliography{sample-base}

\end{document}
\endinput