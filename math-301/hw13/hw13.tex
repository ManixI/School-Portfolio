
\documentclass[sigconf]{article}
\usepackage{savesym}
\usepackage{amsmath}
\usepackage{graphicx}
\usepackage{hyperref}
\usepackage{listings}
\usepackage{helvet}
\usepackage{ragged2e}
\usepackage{tikz}
\usepackage{titlesec}
\usepackage{courier}
\usepackage{pdfpages}
\usepackage{txfonts}
\usepackage{mathrsfs}
\usepackage{realhats}
\usepackage{array}
%\usepackage{MnSymbol}


\usetikzlibrary{er,positioning, arrows.meta}

\lstset{basicstyle=\footnotesize\ttfamily, language=Java}



\graphicspath{ {./images/} }
%%
%% \BibTeX command to typeset BibTeX logo in the docs
%\AtBeginDocument{%
%  \providecommand\BibTeX{{%
%    Bib\TeX}}}

\titleformat{\section}
  {\normalfont\Large\bfseries}
  {\thesection}
  {1em}
  {}
  [{\titlerule[0.8pt]}]

\titleformat{\subsection}
  {\normalfont\Large\bfseries}
  {\thesection}
  {1em}
  {}
  [{\titlerule[0.3pt]}]

  \titleformat{\title}
  {\normalfont\Large\bfseries}
  {\thesection}
  {1em}
  {{\titlerule[0.8pt]}}
  [{\titlerule[0.8pt]}]


\title{
  %\line(1,0){250} \\
  \textbf{Homework 13} \\
  %\large \textbf{A game review} \\
  %\line(1,0){250}
  }
\author{ 
  Ian Manix
  }


%\renewcommand*\contentsname{Table of Contents}

\begin{document}

%%
%% The "title" command has an optional parameter,
%% allowing the author to define a "short title" to be used in page headers.



\maketitle

%\clearpage



%\clearpage
%\addcontentsline{toc}{section}{Concept}
\begin{enumerate}
  \item Addition for $\mathbb{Z}_5$:\\
    \begin{tabular}{c|ccccc}
      + & 0 & 1 & 2 & 3 & 4 \\
      \hline
      0 & 0 & 1 & 2 & 3 & 4 \\
      1 & 1 & 2 & 3 & 4 & 0 \\
      2 & 2 & 3 & 4 & 0 & 1 \\
      3 & 3 & 4 & 5 & 0 & 1 \\
      4 & 4 & 0 & 1 & 2 & 3 \\
    \end{tabular}\\\\
    Multiplication for $\mathbb{Z}_5$:\\
    \begin{tabular}{c|ccccc}
      X & 0 & 1 & 2 & 3 & 4 \\
      \hline
      0 & 0 & 0 & 0 & 0 & 0 \\
      1 & 0 & 1 & 2 & 3 & 4 \\
      2 & 0 & 2 & 4 & 1 & 3 \\
      3 & 0 & 3 & 1 & 4 & 2 \\
      4 & 0 & 4 & 3 & 2 & 1 \\
    \end{tabular}
  \item $\mathbb{Z}_5$ is a field because it satisfies all the conditions of a field.\\
      It is closed under addition and subtraction because $\mathbb{Z}_5\subset\mathbb{Z}$ and $\mathbb{Z}$ is closed under addition and multiplication.\\
      It has an element 0 where for all $a\in\mathbb{Z}_5$, $a+0=a$ as shown in the above addition table.\\
      It has identity because there exists the identity element 1 where for all $a\in\mathbb{Z}_5$, $1a=a$.\\
      Every element of $\mathbb{Z}_5$ is a unit (that is $\forall a, \exists a^{-1}: a,a^{-1}\in\mathbb{Z}_5, a\neq0, aa^{-1}=1$) as shown in the above multiplication table.
      \clearpage
  \item Addition for $\mathbb{Z}_4$:\\
    \begin{tabular}{c|cccc}
      + & 0 & 1 & 2 & 3 \\
      \hline
      0 & 0 & 1 & 2 & 3 \\
      1 & 1 & 2 & 3 & 0 \\
      2 & 2 & 3 & 0 & 1 \\
      3 & 0 & 0 & 1 & 2 \\
    \end{tabular}\\\\
    Multiplication for $\mathbb{Z}_4$:\\
    \begin{tabular}{c|cccc}
      X & 0 & 1 & 2 & 3 \\
      \hline
      0 & 0 & 0 & 0 & 0 \\
      1 & 0 & 1 & 2 & 3 \\
      2 & 0 & 2 & 0 & 2 \\
      3 & 0 & 3 & 2 & 1 \\
    \end{tabular}
  \item $\mathbb{Z}_4$ is not a field because not every value is a unit. The identity value is 1, and there is no value $a\in\mathbb{Z}_4$ such that $2a=1$.
  \item 2-dimensional vectors whose elements are those of $\mathbb{Z}_5$ form a vector space over $\mathbb{Z}_5$ because it satisfies the following axioms.\\
      Vector addition using these elements is closed because addition is closed in $\mathbb{Z}_5$.\\
      It is commutative because $\mathbb{Z}_5$ is commutative.\\
      It is associative because $\mathbb{Z}_5$ is associative.\\
      It is closed under multiplication because $\mathbb{Z}_5$ is.\\
      Every element is a unit because this is also true of $\mathbb{Z}_5$.\\
      Scalar and field multiplication are compatible because multiplication is associative in $\mathbb{Z}_5$.\\
      There exists an additive inverse for every element because this is also true in $\mathbb{Z}_5$.\\
      There is an identity element 1 because there is one in $\mathbb{Z}_5$.\\
      Scalar multiplication with vector addition and field addition are distributive because both multiplication and addition are distributive in $\mathbb{Z}_5$.
\end{enumerate}



% $\mathscr{P}$
% $\hat[ash]{a}$
\subsection*{Reference:}
Hammack, R. H. (2009). Book of Proof.\\ https://orion.math.iastate.edu/jdhsmith/class/BookOfProof.pdf
\\
\\
JUDSON, T. (2023). Abstract algebra: Theory and applications. ORTHOGONAL PUBLISHING L3C.

%%
%% The next two lines define the bibliography style to be used, and
%% the bibliography file.
\bibliographystyle{ACM-Reference-Format}
\bibliography{sample-base}

\end{document}
\endinput