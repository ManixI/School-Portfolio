
\documentclass[sigconf]{article}
\usepackage{savesym}
\usepackage{amsmath}
\usepackage{graphicx}
\usepackage{hyperref}
\usepackage{listings}
\usepackage{helvet}
\usepackage{ragged2e}
\usepackage{tikz}
\usepackage{titlesec}
\usepackage{courier}
\usepackage{pdfpages}
\usepackage{txfonts}
\usepackage{mathrsfs}
\usepackage{realhats}
%\usepackage{MnSymbol}


\usetikzlibrary{er,positioning, arrows.meta}

\lstset{basicstyle=\footnotesize\ttfamily, language=Java}



\graphicspath{ {./images/} }
%%
%% \BibTeX command to typeset BibTeX logo in the docs
%\AtBeginDocument{%
%  \providecommand\BibTeX{{%
%    Bib\TeX}}}

\titleformat{\section}
  {\normalfont\Large\bfseries}
  {\thesection}
  {1em}
  {}
  [{\titlerule[0.8pt]}]

\titleformat{\subsection}
  {\normalfont\Large\bfseries}
  {\thesection}
  {1em}
  {}
  [{\titlerule[0.3pt]}]

  \titleformat{\title}
  {\normalfont\Large\bfseries}
  {\thesection}
  {1em}
  {{\titlerule[0.8pt]}}
  [{\titlerule[0.8pt]}]


\title{
  %\line(1,0){250} \\
  \textbf{Portfolio 2} \\
  %\large \textbf{A game review} \\
  %\line(1,0){250}
  }
\author{ 
  Ian Manix
  }


%\renewcommand*\contentsname{Table of Contents}

\begin{document}

%%
%% The "title" command has an optional parameter,
%% allowing the author to define a "short title" to be used in page headers.



\maketitle

%\clearpage



%\clearpage
%\addcontentsline{toc}{section}{Concept}
\begin{enumerate}
  \item Consider a right triangle with hypotenuse of length c, and sides of length $a$ and $b$. Prove that this triangle is isosceles if and only if its area is $\frac{c^2}{4}$.\\
        Suppose there is a right isosceles triangle.\\
        Then it's legs $a$ and $b$ must be of equal length x.\\
        So $c^2=x^2+x^2$, or $c=\sqrt{2x^2}$.\\
        It's area can be written as $A=x*x*\frac{1}{2}=\frac{x^2}{2}$.\\
        Then $c^2=a^2+b^2=2x^2$ by the Pythagorean theorem.\\
        So $\frac{c^2}{2}=x^2$, then substituting $x^2$ in the formula for the area of the triangle we get $A=\frac{1}{2}*\frac{c^2}{2}=\frac{c^2}{4}$.\\
        Then suppose there is a triangle with area $A=\frac{c^2}{4}$.\\
        %The area of the triangle can be written as $\frac{c^2}{4}=\frac{1}{2}\frac{c}{2}b+\frac{1}{2}\frac{c}{2}a$.\\

        %The area of this triangle can be written as $A=\frac{c^2}{4}=\frac{1}{2}ab$ where $a$ and $b$ are the legs of the triangle.\\
        %So multiply both sides by 2 as we get $\frac{c^2}{2}=ab$.\\
        %Because it is a triangle we also know that $c^2=a^2+b^2$.\\% which can be written as $a=\sqrt{c^2-b^2}$ or $ab=b\sqrt{c^2-b^2}$.\\
        %Substituting the value of the area we get $\frac{c^2}{2}=b\sqrt{c^2-b^2}$.\\
        %Squaring both sides gives us $\frac{c^4}{4}=b^2(c^2-b^2)$
        %Dividing by 2 we can substitute $c^2$ to get $ab=\frac{a^2+b^2}{2}$.\\
        %This becomes $2ab=a^2+b^2$.\\
        %$2b=a+\frac{b^2}{a}$.\\
        %$2ab-b^2=b(2a-b)=a^2$.\\


  \item For all integers $a$ and $b$, $a\equiv b\mod 5$ of and only if $(a+3b)\equiv(3a+b)\mod 10$.\\
        Let $a\equiv b\mod 5$.\\
        Then $b-a=5n$ where $n\in\mathbb{Z}$ by definition of divisibility.\\
        So $(a+3b)-(3a+b)=-2a+2b=2(b-a)$.\\
        Thus $10n=(a+3b)-(3a+b)$, showing $10|(a+3b)-(3a+b)$.\\
        Therefor $(a+3b)\equiv(3a+b)\mod 10$.\\
        Then let $(a+3b)\equiv(3a+b)\mod 10$.\\
        This means $2(b-a)=10n$ where $n\in\mathbb{Z}$.\\
        Then it follows that $b-a=5n$, therefor $a\equiv b\mod 5$.
        Thus because $a\equiv b\mod 5$ implies $(a+3b)\equiv(3a+b)\mod 10$ and $(a+3b)\equiv(3a+b)\mod 10$ implies $a\equiv b\mod 5$, $a\equiv b\mod 5$ if and only if $(a+3b)\equiv(3a+b)\mod 10$.

  \item Let $A,B,C$ be nonempty sets. Prove that if $A\cup C=B\cup C$, and $A\cap C=B\cap C$, then $A=B$.\\
        Let $A,B,C$ be nonempty sets where $A\cup C=B\cup C$, and $A\cap C=B\cap C$.\\
        Then let $x\in A$.\\
        So because $x\in A\cup C$ and $A\cup C=B\cup C$, then $x\in B$ or $x\in C$.\\
        If $x\in C$, then $x\in A\cap C$ by definition of intersection.\\
        Thus $x\in B\cap C$, so $x\in B$ by definition of intersection.\\
        So $x\in B$ for either case, and therefor $A\subseteq B$.\\
        Then let $y\in B$.\\
        So because $y\in B\cup C$ and $A\cup C=B\cup C$, then $y\in A$ or $y\in C$.\\
        If $y\in C$, then $y\in B\cap C$ by definition of intersection.\\
        Thus $y\in A\cap C$, so $y\in A$ by definition of intersection.\\
        So $y\in A$ for either case, and therefor $B\subseteq A$.\\
        Therefor because $A\subseteq B$ and $B\subseteq A$, $A=B$.

  \item Let $U$ be a universe of sets, and let $X,Y$ be any subsets of $U$. Define a relations $X~Y$ if and only if there exists a bijection $f:X\rightarrow Y$. Prove that ~ is an equivalence relation on subsets of $U$. \\
        First, for any subset of $X$ of $U$, the identity function $f:X\rightarrow X$, defined by $f(x)=x$ for all $x\in X$, is a bijection from X to itself.\\
        Therefor $X~X$, meaning it is reflexive.\\
        Next, suppose there exists a bijection $f:X\rightarrow Y$.\\
        There then must be an inverse $f^{-1}:Y\rightarrow X$.\\
        So there exists a bijection from $X$ to $Y$, so $Y~X$ whenever $X~Y$.\\
        meaning it is symmetrical.\\
        Finally, suppose there exists the bijection $f:X\rightarrow Y$ and the bijection $g:Y\rightarrow Z$.\\
        So the composition of $f\circ f:X\rightarrow Z$ is a bijection from $X$ to $Z$.\\
        Therefor $X~Z$ whenever $X~Y$ and $Y~Z$, thus it is transitive.\\
        Because the bijection is reflexive, symmetrical, and transitive $~$ is an equivalence relation onto subsets of $U$.
\end{enumerate}

% $\mathscr{P}$
% $\hat[ash]{a}$

Hammack, R. H. (2009). Book of Proof.\\ https://orion.math.iastate.edu/jdhsmith/class/BookOfProof.pdf
\\
\\
JUDSON, T. (2023). Abstract algebra: Theory and applications. ORTHOGONAL PUBLISHING L3C. 
%%
%% The next two lines define the bibliography style to be used, and
%% the bibliography file.
\bibliographystyle{ACM-Reference-Format}
\bibliography{sample-base}

\end{document}
\endinput