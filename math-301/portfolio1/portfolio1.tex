
\documentclass[sigconf]{article}
\usepackage{savesym}
\usepackage{amsmath}
\usepackage{graphicx}
\usepackage{hyperref}
\usepackage{listings}
\usepackage{helvet}
\usepackage{ragged2e}
\usepackage{tikz}
\usepackage{titlesec}
\usepackage{courier}
\usepackage{pdfpages}
\usepackage{txfonts}
\usepackage{mathrsfs}
\usepackage{realhats}
%\usepackage{MnSymbol}


\usetikzlibrary{er,positioning, arrows.meta}

\lstset{basicstyle=\footnotesize\ttfamily, language=Java}



\graphicspath{ {./images/} }
%%
%% \BibTeX command to typeset BibTeX logo in the docs
%\AtBeginDocument{%
%  \providecommand\BibTeX{{%
%    Bib\TeX}}}

\titleformat{\section}
  {\normalfont\Large\bfseries}
  {\thesection}
  {1em}
  {}
  [{\titlerule[0.8pt]}]

\titleformat{\subsection}
  {\normalfont\Large\bfseries}
  {\thesection}
  {1em}
  {}
  [{\titlerule[0.3pt]}]

  \titleformat{\title}
  {\normalfont\Large\bfseries}
  {\thesection}
  {1em}
  {{\titlerule[0.8pt]}}
  [{\titlerule[0.8pt]}]


\title{
  %\line(1,0){250} \\
  \textbf{Portfolio 1} \\
  %\large \textbf{A game review} \\
  %\line(1,0){250}
  }
\author{ 
  Ian Manix
  }


%\renewcommand*\contentsname{Table of Contents}

\begin{document}

%%
%% The "title" command has an optional parameter,
%% allowing the author to define a "short title" to be used in page headers.



\maketitle

%\clearpage



%\clearpage
%\addcontentsline{toc}{section}{Concept}
\section*{Chapter 8}
\begin{enumerate}
  \item Proposition: let $m$ be an integer. Prove that if $m$ is odd then $\frac{m^4-1}{4}$ is even.\\
        Suppose $m$ is an odd integer.\\
        Then $m=2n+1$ by definition of an odd number.\\
        Thus $\frac{m^4-1}{4}=\frac{(2n+1)^4-1}{4}=\frac{16n^4+32n^3+24n^2+8n}{4}=4n^4+8n^3+6n^2+2n$.\\
        Then $4n^4+8n^3+6n^2+2n=2(2n^4+4n^3+3n^2+n)=2k$ where $k=2n^4+4n^3+3n^2+1$ and $k\in\mathbb{N}$.\\
        Thus if $m$ is odd, $\frac{m^4-1}{4}$ must be even.

  \item \begin{enumerate}
      \item Proposition: prove that for all $p,q\in\mathbb{Q}$ with $p<q$, there exists $x\in\mathbb{Q}$ such that $p<x<q$.\\
            Suppose for all $p,q\in\mathbb{Q}$, $p<q$.\\
            Then there exists $x$ such that $x=\frac{p+q}{2}$.\\
            So because $x$ is the arithmetic mean of $p$ and $q$, $p<x<q$.\\
            Thus for all$p,q\in\mathbb{Q}$ with $p<q$, there exists $x\in\mathbb{Q}$ such that $p<x<q$.

      \item Proposition: prove that there are infinitely many rational numbers between 0 and 1.\\
            Suppose we have the set $\{\frac{1}{x}:x\in\mathbb{N}$ and $x\neq1\}$.\\
            All the numbers in this set are rational by the definition of a rational number $\mathbb{Q}=\{x:x=\frac{m}{n}$, where $m,n\mathbb{Z}$ and $n\neq0\}$.\\
            Then the set of all integers $\mathbb{N}$ is infinite by definition.\\
            Therefor the set $\{\frac{1}{x}:x\in\mathbb{N}$ and $x\neq1\}$ contains infinitely many rational numbers.\\
            Then the set $\{\frac{1}{x}:x\in\mathbb{N}$ and $x\neq1\}$ will always be less then 1 because it's largest value is $\frac{1}{2}$, with every subsequent value getting smaller.\\
            And it will always be greater then 1 because there is no integer $x$ where $\frac{1}{x}\leq0$.\\
            Therefore the set $\{\frac{1}{x}:x\in\mathbb{N}$ and $x\neq1\}$ describes an infinite number of rational numbers between 0 and 1.

    \end{enumerate}
  \item Let $a,b$ be natural numbers such $a^2=b^3$. Prove that if $a$ is even, then 4 divides $a$.\\
        Suppose 4 dose not divide $a$.\\
        Then $a=2b+2c+1$, where $b,c\in\mathbb{Z}$ and $c=0$ or $c=1$.\\
        That is to say $a=4d+1$ or $a=4d+3$ for $d\in\mathbb{Z}$.\\
        So $b^3(2b+2c+1)^2=4b^2+4c^2+8bc+4b+4c+1=4(b(b+c)+c(b+c)+b+c)+1=4m+1$ where $m\in\mathbb{Z}$.\\
        Then because a perfect cube is either 1 mod 4 or -1 mod 4, $b^3$ cannot be in the form $4m+3$.\\
        So $b^3=2(2m)+1$, meaning $b^3$ is odd by definition of an odd number.\\
        Then $b$ is odd because an odd number cubed is odd.\\
        So $(2b+2c+1)^2=(2x+1)^3$ where $x\in\mathbb{Z}$.\\
        Thus $4b^2+4c^2+8kr+1=8x^3+12x^2+6x+1$.\\
        Then subtract 1 from each side to get $4b^2+4c^2+8bc=8x^3+12x^2+6x$ or $2(2b^2+2c^2+4bc)=2(4x^3+6x^2+3x)$.\\
        Thus $2y=2z$ where $y,z\in\mathbb{Z}$.\\
        Thus both sides are even, but this contradicts the fact that $a=4d+1$ or $a=4d+3$, thus $a$ cannot be in this form.\\
        Thus assuming $a^2=b^3$, if $a$ is even then 4 bust divide $a$.




  \item Prove that if a natural number is divisible by 4, then it is a difference in perfect squares.
        Suppose $x\in\mathbb{Z}$ and $x$ is divisible by 4.\\
        Then $x=4a$ by definition of divisibility.\\
        Then $x=4k+1-1+k^2-k^2=(k^2+2k+1)-(k^2-2k+1)=(k+1)^2-(k-1)^2=m^2-n^2$ where $m,n\in\mathbb{Z}$.\\
        Therefor if $x$ is divisible by 4 then $x=m^2-n^2$, meaning if $x$ is divisible by 4 it can be written as the difference of two perfect squares.
\end{enumerate}

% $\mathscr{P}$
% $\hat[ash]{a}$

Hammack, R. H. (2009). Book of Proof.\\ https://orion.math.iastate.edu/jdhsmith/class/BookOfProof.pdf

%%
%% The next two lines define the bibliography style to be used, and
%% the bibliography file.
\bibliographystyle{ACM-Reference-Format}
\bibliography{sample-base}

\end{document}
\endinput