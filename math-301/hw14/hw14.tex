
\documentclass[sigconf]{article}
\usepackage{savesym}
\usepackage{amsmath}
\usepackage{graphicx}
\usepackage{hyperref}
\usepackage{listings}
\usepackage{helvet}
\usepackage{ragged2e}
\usepackage{tikz}
\usepackage{titlesec}
\usepackage{courier}
\usepackage{pdfpages}
\usepackage{txfonts}
\usepackage{mathrsfs}
\usepackage{realhats}
\usepackage{array}
%\usepackage{MnSymbol}


\usetikzlibrary{er,positioning, arrows.meta}

\lstset{basicstyle=\footnotesize\ttfamily, language=Java}



\graphicspath{ {./images/} }
%%
%% \BibTeX command to typeset BibTeX logo in the docs
%\AtBeginDocument{%
%  \providecommand\BibTeX{{%
%    Bib\TeX}}}

\titleformat{\section}
  {\normalfont\Large\bfseries}
  {\thesection}
  {1em}
  {}
  [{\titlerule[0.8pt]}]

\titleformat{\subsection}
  {\normalfont\Large\bfseries}
  {\thesection}
  {1em}
  {}
  [{\titlerule[0.3pt]}]

  \titleformat{\title}
  {\normalfont\Large\bfseries}
  {\thesection}
  {1em}
  {{\titlerule[0.8pt]}}
  [{\titlerule[0.8pt]}]


\title{
  %\line(1,0){250} \\
  \textbf{Homework 14} \\
  %\large \textbf{A game review} \\
  %\line(1,0){250}
  }
\author{ 
  Ian Manix
  }


%\renewcommand*\contentsname{Table of Contents}

\begin{document}

%%
%% The "title" command has an optional parameter,
%% allowing the author to define a "short title" to be used in page headers.



\maketitle

%\clearpage



%\clearpage
%\addcontentsline{toc}{section}{Concept}
\subsection*{13.2}
\begin{itemize}
  \item[3.] Prove that $\lim_{x\to0}(x+2)=2$.\\
            Given $\epsilon >0$.\\
            Then let $\delta=\epsilon$.\\
            Thus $0<|x+2-2|=|x-0|<\delta$.\\
            Therefor $\lim_{x\to0}(x+2)=2$ by definition 13.2.\\

  \item[5.] Prove that $\lim_{x\to3}(x^2-2)=7$.\\
            Given $\epsilon>0$.\\
            Then let $\delta=\epsilon$.\\
            So $|(x^2-2)-7|=|x^2-9|=|(x+3)(x-3)|$.\\
            If $|x-3|\leq1$ then $|x+3|\leq|x-3|+|6|\leq7$.\\
            This means that $|x-3|*|x-3|<7|x-3|$.\\
            Then take $\delta$ to be smaller then both 1 and $\frac{\epsilon}{7}$.\\
            So $0<|x-3|<\delta\implies|(x^2-2)-7|<7|x-3|<7\delta<s\frac{\epsilon}{7}=\epsilon$.\\
            Thus $\lim_{x\to3}(x^2-2)=7$.
\end{itemize}

\subsection*{13.4}
\begin{itemize}
  \item[6.] Prove the squeeze theorem: Suppose $g(x)\leq f(x)\leq h(x)$ for all $x\in\mathbb{R}$ satisfying $0<|x-c|<\delta$ for some $\delta>0$. If $\lim_{x\to c}g(x)=\lim_{x\to c}h(x)$=L, Then $\lim_{x\to c}f(x) = L$.\\ 
            Suppose $g(x)\leq f(x)\leq h(x)$ for all $x\in\mathbb{R}$ satisfying $0<|x-c|<\delta$ for some $\delta>0$ and $\lim_{x\to c}g(x)=\lim_{x\to c}h(x)=L$.\\
            Then because $0|x-c|<\delta$, $0<f(x)-g(x)<h(x)-g(x)$.\\
            So $\lim_{x\to c}(f(x)-g(x)=\lim_{x\to c}f(x)-\lim_{x\to c}g(x)=\lim_{x\to c}f(x)-L=0$ by theorem 13.6.\\
            Also $\lim_{x\to c}(h(x)-g(x))=\lim_{x\to c}h(x)-\lim_{x\to c}g(x)=L-L=0$.\\
            So because $\lim_{x\to c}(f(x)-g(x)=0$ and $\lim_{x\to c}f(x)-L=0$, $\lim_{x\to c}f(x)=L$.


\end{itemize}

\subsection*{13.7}
\begin{itemize}
  \item[2.] Prove that $\{5+\frac{2}{n^2}\}$ converges to 5.\\
            Suppose $\epsilon>0$.\\
            The chose an integer $N>\frac{2}{\epsilon}$ so that $\frac{2}{N^2}>\epsilon$.\\
            Then if $n>N$, we have $|a_n-5|=|(5+\frac{2}{n^2})-5|=\frac{2}{n^2}<\frac{2}{N^2}<\epsilon$.\\
            By definition 12.5 the sequence $\{5+\frac{2}{n^2}\}$ converges to 5.
  \item[4.] Prove that $\{1-\frac{1}{2^n}\}$ converges to 1.\\
            Suppose $\epsilon>0$.\\
            The chose an integer $N>\frac{1}{\epsilon}$ so that $\frac{1}{2^N}>\epsilon$.\\
            Then if $n>N$, we have $|a_n-1|=|(1-\frac{1}{2^n})-1|=\frac{1}{2^n}<\frac{1}{2^N}<\epsilon$.\\
            By definition 12.5 the sequence $\{1-\frac{1}{2^n}\}$ converges to 1.
  \item[10.] Prove that if $\{a_n\}$ converges to L and $\{b_n\}$ converges to M, then the sequence $\{a_n+b_n\}$ converges to $L+M$.\\
            Let $\{a_n\}$ converge to L and $\{b_m\}$ converge to M.\\
            Then there must exist integers $N_1$ and $N_2$ such that for all $n<N_1$, $|a_n-L|<\frac{\epsilon}{2}$ and for all $n<N_2$, $|b_n-M|<\frac{\epsilon}{2}$.\\
            Now let $N=Max(N_1,N_2)$.\\
            So $|(a_n+b_n)-(L+M)|=|(a_n-L)+(b_n-M)|$.\\
            Then $|(a_n-L)+(b_n-M)|\leq|a_n-L|+|b_n-M|\leq\frac{\epsilon}{2}+\frac{\epsilon}{2}=2$.\\
            Therefor because $|(a_n-L)+(b_n-M)|<\epsilon$, $\{a_n+b_n\}$ converges to $L+M$.
\end{itemize}



% $\mathscr{P}$
% $\hat[ash]{a}$
\subsection*{Reference:}
Hammack, R. H. (2009). Book of Proof.\\ https://orion.math.iastate.edu/jdhsmith/class/BookOfProof.pdf
\\
\\
JUDSON, T. (2023). Abstract algebra: Theory and applications. ORTHOGONAL PUBLISHING L3C.

%%
%% The next two lines define the bibliography style to be used, and
%% the bibliography file.
\bibliographystyle{ACM-Reference-Format}
\bibliography{sample-base}

\end{document}
\endinput