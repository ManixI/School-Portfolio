
\documentclass[sigconf]{article}
\usepackage{graphicx}
\usepackage{hyperref}
\usepackage{listings}
\usepackage{helvet}
\usepackage{ragged2e}
\usepackage{tikz}
\usepackage{titlesec}
\usepackage{courier}
\usepackage{pdfpages}
\usepackage{txfonts}
\usepackage{array}

\usetikzlibrary{er,positioning, arrows.meta}

\lstset{basicstyle=\footnotesize\ttfamily, language=Java}



\graphicspath{ {./images/} }
%%
%% \BibTeX command to typeset BibTeX logo in the docs
%\AtBeginDocument{%
%  \providecommand\BibTeX{{%
%    Bib\TeX}}}

\titleformat{\section}
  {\normalfont\Large\bfseries}
  {\thesection}
  {1em}
  {}
  [{\titlerule[0.8pt]}]

\titleformat{\subsection}
  {\normalfont\Large\bfseries}
  {\thesection}
  {1em}
  {}
  [{\titlerule[0.3pt]}]
{}
  \titleformat{\title}
  {\normalfont\Large\bfseries}
  {\thesection}
  {1em}
  {{\titlerule[0.8pt]}}
  [{\titlerule[0.8pt]}]


\title{
  %\line(1,0){250} \\
  \textbf{Preparation 12} \\
  %\large \textbf{A game review} \\
  %\line(1,0){250}
  }
\author{ 
  Ian Manix
  }


%\renewcommand*\contentsname{Table of Contents}

\begin{document}

%%
%% The "title" command has an optional parameter,
%% allowing the author to define a "short title" to be used in page headers.



\maketitle

%\clearpage



%\clearpage
%\addcontentsline{toc}{section}{Concept}

\begin{enumerate}
  \item The law of composition must be associative, there must be an identity element, each element has an inverse element, each element must be unique, and it must be a set with a composition law
  \item $\mathbb{Z}_4$: \\
    \setlength\extrarowheight{3pt}
    \begin{tabular}{c | c c c c c}
      X & 0 & 1 & 2 & 3  \\
      \hline
      0 & 0 & 0 & 0 & 0 \\
      1 & 0 & 1 & 2 & 3 \\
      2 & 0 & 2 & 0 & 2 \\
      3 & 0 & 3 & 2 & 1 \\
    \end{tabular}\\
    $\mathbb{Z}_3$: \\
    \setlength\extrarowheight{3pt}
    \begin{tabular}{c | c c c c c}
      X & 0 & 1 & 2  \\
      \hline
      0 & 0 & 0 & 0 \\
      1 & 0 & 1 & 2 \\
      2 & 0 & 2 & 1 \\
    \end{tabular}\\
  \item $\mathbb{Z}_4$ and $\mathbb{Z}_3$ are groups, the values in the Cayley tables are not. This is because the values in the Cayley tables are not all unique.
\end{enumerate}

%%
%% The next two lines define the bibliography style to be used, and
%% the bibliography file.
\bibliographystyle{ACM-Reference-Format}
\bibliography{sample-base}

\end{document}
\endinput