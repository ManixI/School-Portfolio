
\documentclass[sigconf]{article}
\usepackage{savesym}
\usepackage{amsmath}
\usepackage{graphicx}
\usepackage{hyperref}
\usepackage{listings}
\usepackage{helvet}
\usepackage{ragged2e}
\usepackage{tikz}
\usepackage{titlesec}
\usepackage{courier}
\usepackage{pdfpages}
\usepackage{txfonts}
\usepackage{mathrsfs}
\usepackage{realhats}
%\usepackage{MnSymbol}


\usetikzlibrary{er,positioning, arrows.meta}

\lstset{basicstyle=\footnotesize\ttfamily, language=Java}



\graphicspath{ {./images/} }
%%
%% \BibTeX command to typeset BibTeX logo in the docs
%\AtBeginDocument{%
%  \providecommand\BibTeX{{%
%    Bib\TeX}}}

\titleformat{\section}
  {\normalfont\Large\bfseries}
  {\thesection}
  {1em}
  {}
  [{\titlerule[0.8pt]}]

\titleformat{\subsection}
  {\normalfont\Large\bfseries}
  {\thesection}
  {1em}
  {}
  [{\titlerule[0.3pt]}]

  \titleformat{\title}
  {\normalfont\Large\bfseries}
  {\thesection}
  {1em}
  {{\titlerule[0.8pt]}}
  [{\titlerule[0.8pt]}]


\title{
  %\line(1,0){250} \\
  \textbf{Portfolio 3} \\
  %\large \textbf{A game review} \\
  %\line(1,0){250}
  }
\author{ 
  Ian Manix
  }


%\renewcommand*\contentsname{Table of Contents}

\begin{document}

%%
%% The "title" command has an optional parameter,
%% allowing the author to define a "short title" to be used in page headers.



\maketitle

%\clearpage
%\addcontentsline{toc}{section}{Concept}
\begin{enumerate}
  \item Prove that polynomials in $\mathbb{Z}_3[i]=\{a+bi:a,b\in\mathbb{Z}$ form a vector space assuming $\mathbb{Z}_3[i]$ is a field.\\
        Let $\mathbb{X}_3[1]$ be a field.\\
        For polynomials in $\mathbb{Z}_3[i]=\{a+bi:a,b\in\mathbb{Z}$ to form a vector space, they must meet 10 axioms.\\
        Given that $\mathbb{Z}_3[i]$ is a field, by definition it must be closed under addition, commutative, associative, there is a 0 element, there is an additive inverse for every element, there is an identity element 1, and it is closed under scalar multiplication.\\
        Then, because multiplication is distributive by definition of a field, it follows that in $\mathbb{Z}_3[i]$, scalar and field multiplication are compatible, that scalar multiplication is distributive in regards to field addition, and that scalar multiplication is distributive in regards to vector addition.\\
        Thus, since all elements of a vector space are satisfied, polynomials in $\mathbb{Z}_3[i]=\{a+bi:a,b\in\mathbb{Z}$ form a vector space.

  \item Find all zeros and fully factor $p(x)=x^5+(1+i)x^4+2ix^3+(2+i)x^2+2+2i$ in $\mathbb{Z}_3[i]$.\\
        $p(0)=0^5+(1+i)0^4+2i0^3+(2+i)0^2+2+2i=2+2i$.\\
        $p(1)=1+(1+i)+2i+(2+i)+2+2i=6+6i\equiv 0$.\\
        $p(2)=32+(1+i)16+2i8+(2+i)4+2+2i=58+38i\equiv 1+2i$.\\
        $p(i)=i+(1+i)+2+(2+i)+2+2i=7+5i\equiv1+2i$.\\
        $p(2i)=32i+(1+i)16+16+8+4i+2+2i=42+54i\equiv0$.\\
        $p(1+i)=12+6i\equiv0$.\\
        $p(2+i)=87+75i\equiv0$.\\
        $p(1+i2)=54+84i\equiv0$.\\
        $p(2+2i)=230+206i\equiv2+2i$.\\
        Therefor the zeros are 1, 2i, (1+i), (1+2i), and (2+i).\\
        Thus the fully factored polynomial is $p(x)=x^5+(1+i)x^4+2ix^3+(2+i)x^2+2+2=(x-1)(x-2i)(x-(1+i))(x-(1+2i))(x-(2+i))$.

  \item Disprove the following. A polynomial in $\mathbb{Z}_n$ has more zeros then it's highest power.\\
        Take the polynomial $f(x)=x^2$ in $\mathbb{Z}_3$.\\
        The values of this are $f(0)=0$, $f(1)=1$, and $f(2)=4\equiv 1$.\\
        Thus $f(x)=x^2$ has 1 zero, which is not greater then it's highest power of $x^2$.\\
        Therefor the statement is false.

  \item Find the limit of $f(x)=\ln(5x^4-x^3-3)$ at $x=1$ assuming $ln(x)$ is continuous.\\
        %First, because $ln(x)$ is continuous, $\lim_{x\to1}\ln(5x^4-x^3-3)$ can be written as $\ln(\lim_{x\to1}5x^4-x^3-3)$ by properties of continuity.\\
        First, $f(1)=\ln5(1)^4-1^3-3=\ln1=0$, so $|\ln(5x^4-x^3-3)-0|<\epsilon$.\\
        And we know that given any $|\epsilon>0$, there exists $\delta>0$ such that for all $x\neq c$, if $|x-c|<\delta$, then $|f(x)-L|<\epsilon$.\\
        Then, because $\ln(x)$ is continuous we can say $|5x^4-x^3-4|<\epsilon$.\\
        Next let $\delta=\frac{\epsilon}{5}$ so:\\
        $|x-1|<\delta=\frac{\epsilon}{5}$\\
        $-\frac{\epsilon}{5}<x-1<\frac{\epsilon}{5}$\\
        $1-\frac{\epsilon}{5}<x<1+\frac{\epsilon}{5}$.\\
        Then take $|5x^4-x^3-4|=5(1+\frac{\epsilon}{5})^4-(1+\frac{\epsilon}{5})^3-4|$.\\
        Because both $(1+\frac{\epsilon}{5})^4$ and $(1+\frac{\epsilon}{5})^3$ approach 1 we can say $|5x^4-x^3-4|<\epsilon$.\\
        Therefor we can verify that $\lim_{x\to1}\ln(5x^4-x^3-3)$ is indeed 0.

        %So let $|x-1|<\delta=\frac{\sqrt{\sqrt20\epsilon+121}-1}{\sqrt{10}}-1$.\\


        %$|5x^4-x^3-3-1|<\epsilon$.\\
        %$|x^3(5x-1)|<\epsilon+4$.\\
        %$|5x-1|<\frac{\epsilon+4}{x^3}$.\\
        %$|x-\frac{1}{5}|<\frac{\epsilon+4}{5x^3}$.\\
        %$|x-1|<\frac{\epsilon+4}{5x^3}-\frac{4}{5}$.\\
        %Then $|x-1|<\delta=\frac{\sqrt{\sqrt20\epsilon+121}-1}{\sqrt{10}}-1$ implies $|\ln(5x^4-x^3-3)-L|<\epsilon$ for $\epsilon>0$.\\

        %Because $ln(x)$ is continuous, $\lim_{x\to1}\ln(5x^4-x^3-3)$ can be written as $\ln(\lim_{x\to1}5x^4-x^3-3)$

\end{enumerate}

% $\mathscr{P}$
% $\hat[ash]{a}$

Hammack, R. H. (2009). Book of Proof.\\ https://orion.math.iastate.edu/jdhsmith/class/BookOfProof.pdf
\\
\\
JUDSON, T. (2023). Abstract algebra: Theory and applications. ORTHOGONAL PUBLISHING L3C. 
%%
%% The next two lines define the bibliography style to be used, and
%% the bibliography file.
\bibliographystyle{ACM-Reference-Format}
\bibliography{sample-base}

\end{document}
\endinput